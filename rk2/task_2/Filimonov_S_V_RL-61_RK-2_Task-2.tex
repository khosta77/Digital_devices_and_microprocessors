\documentclass{bmstu}
\usepackage{multirow}
%\usepackage{karnaugh-map}
\usepackage{tikz}
\usepackage{float}
\usepackage{fancyhdr}
\usepackage{graphicx}
\usepackage{titlesec}
\usepackage{titletoc}
\usepackage{listings}
\usepackage{appendix}
\usepackage{bm, amsmath, amsfonts}
\usepackage{multirow}
\usepackage{hyperref}
\usepackage{subfig}
\usepackage{url}
\usepackage{multirow}
\usepackage{array} 
\usepackage{wrapfig} % in preamble

% set the code style
\RequirePackage{listings}
\RequirePackage{xcolor}
\definecolor{dkgreen}{rgb}{0,0.6,0}
\definecolor{gray}{rgb}{0.5,0.5,0.5}
\definecolor{mauve}{rgb}{0.58,0,0.82}
\lstset{
	numbers=left,  
	frame=tb,
	aboveskip=3mm,
	belowskip=3mm,
	showstringspaces=false,
	columns=flexible,
	framerule=1pt,
	rulecolor=\color{gray!35},
	backgroundcolor=\color{gray!5},
	basicstyle={\ttfamily},
	numberstyle=\tiny\color{gray},
	keywordstyle=\color{blue},
	commentstyle=\color{dkgreen},
	stringstyle=\color{mauve},
	breaklines=true,
	breakatwhitespace=true,
	tabsize=3
}

\usetikzlibrary{karnaugh}

\begin{document}
	\makereporttitle
	{Радиоэлектроника и лазерная техника (РЛ)} % Название факультета
	{Технология приборостроения (РЛ6)} % Название кафедры
	{Рубежный контроль №2} % Название работы (в дат. падеже)
	{Цифровые устройства и микропроцессоры} % Название курса (необязательный аргумент)
	{Задание 2} % Тема работы
	{} % Номер варианта (необязательный аргумент)
	{Филимонов~С.~В./РЛ6-61} % Номер группы/ФИО студента (если авторов несколько, их необходимо разделить запятой)
	{Семеренко~Д.~А.} % ФИО преподавателя
	
	\textbf{Задание:} Написать программу для МК, в которой длительность свечения светодиода зависит от величины напряжения на входе
	2-ого канала АЦП, частота сигнала определяется величиной аналогового сигнала на входе бого канала АЦІ. Частота преобразования АЦП 10 Гц.
	
	Идея была считывать сигнал с 2-ух энкодеров и подставлять их значения в параметры таймера для генерации ШИМ.
	
	\textbf{Код:}
	
	\begingroup
	\fontsize{12pt}{12pt}\selectfont
	\begin{lstlisting}[language=C]
#include "../system/include/cmsis/stm32f4xx.h"

#define PIN_ANALOG_READ_1 1
#define PIN_ANALOG_READ_2 2
#define PIN_LED_GREAN 12
#define S 1

#define SIZE 2
uint32_t adc1[SIZE];

void GPIO_init() {
	RCC->AHB1ENR |= RCC_AHB1ENR_GPIOAEN;
	GPIOA->MODER |= (GPIO_MODER_MODE0_0 | GPIO_MODER_MODE0_1 | GPIO_MODER_MODE1_0 | GPIO_MODER_MODE1_1); 
}

void DMA2_Stream4_IRQHandler(void) {
	if((DMA2->HISR & (DMA_HISR_TCIF4 | DMA_HISR_HTIF4)) == (DMA_HISR_TCIF4 | DMA_HISR_HTIF4)) {
		DMA2->HIFCR = (DMA_HIFCR_CTCIF4 | DMA_HIFCR_CHTIF4);
	} 
}

void DMA_init() {
	RCC->AHB1ENR |= RCC_AHB1ENR_DMA2EN;
	DMA2_Stream4->PAR |= (uint32_t)&ADC1->DR;
	DMA2_Stream4->M0AR |= (uint32_t)&adc1;
	DMA2_Stream4->NDTR = SIZE;
	DMA2_Stream4->CR = (DMA_SxCR_CIRC | DMA_SxCR_MINC | DMA_SxCR_PSIZE_1 | DMA_SxCR_MSIZE_1 | DMA_SxCR_TCIE);
	NVIC_EnableIRQ(DMA2_Stream4_IRQn);
	NVIC_SetPriority(DMA2_Stream4_IRQn, 4);
	DMA2_Stream4->CR |= DMA_SxCR_EN;
}

void ADC_init(){
	GPIO_init();
	RCC->APB2ENR = RCC_APB2ENR_ADC1EN;
	ADC1->CR1 |= ADC_CR1_SCAN;
	ADC1->CR2 |= (ADC_CR2_ADON | ADC_CR2_CONT | ADC_CR2_EOCS | ADC_CR2_DMA | ADC_CR2_DDS);
	ADC1->SQR3 |= ((0 << 0) | (1 << 5));
	ADC1->SQR1 |= (1 << 20);
	DMA_init();
	ADC1->CR2 |= ADC_CR2_SWSTART;
}

void TIM4_IRQHandler(void) {
	TIM4->SR &= ~TIM_SR_UIF;
	TIM4->CR1 &= ~TIM_CR1_CEN;
	TIM4->PSC = (adc1[0] * 100);
	TIM4->CCR1 = (adc1[1] / 100);
	TIM4->CR1 |= TIM_CR1_CEN;
}

void TIM4_init() {
	RCC->AHB1ENR |= RCC_AHB1ENR_GPIODEN;
	GPIOD->MODER |= (0x2 << (2 * PIN_LED_GREAN));
	GPIOD->AFR[1] |= (0x2 << 16);
	RCC->APB1ENR |= RCC_APB1ENR_TIM4EN;
	TIM4->PSC = 1;
	TIM4->ARR = 100;
	TIM4->CCMR1 |= 0x60;
	TIM4->CCR1 = 1;
	TIM4->CCER |= 0x1;
	TIM4->DIER |= TIM_DIER_UIE;
	NVIC_EnableIRQ(TIM4_IRQn);
	NVIC_SetPriority(TIM4_IRQn, 2);
	TIM4->CR1 |= TIM_CR1_CEN;
}

void init() {
	ADC_init();
	TIM4_init();
}

int main(void) {
	init();
	while(1);
}		
	\end{lstlisting}
	\endgroup
\end{document}

%	На~рисунке~\ref{img:BCD} пример семисегментного индикатора.

%\includeimage
%	{BCD} % Имя файла без расширения (файл должен быть расположен в директории inc/img/)
%	{f} % Обтекание (без обтекания)
%	{H} % Положение рисунка (см. figure из пакета float)
%	{0.25\textwidth} % Ширина рисунка
%	{Семисегментный индикатор} % Подпись рисунка 


%	\begin{lstlisting}[language=C]
	
	%	\end{lstlisting}

%	\begingroup
%	\fontsize{12pt}{12pt}\selectfont
%	\begin{lstlisting}[language=C]
	
	%	\end{lstlisting}
%	\endgroup