\documentclass{bmstu}
\usepackage{multirow}
%\usepackage{karnaugh-map}
\usepackage{tikz}
\usepackage{float}
\usepackage{fancyhdr}
\usepackage{graphicx}
\usepackage{titlesec}
\usepackage{titletoc}
\usepackage{listings}
\usepackage{appendix}
\usepackage{bm, amsmath, amsfonts}
\usepackage{multirow}
\usepackage{hyperref}
\usepackage{subfig}
\usepackage{url}
\usepackage{multirow}
\usepackage{array} 
\usepackage{wrapfig} % in preamble

% set the code style
\RequirePackage{listings}
\RequirePackage{xcolor}
\definecolor{dkgreen}{rgb}{0,0.6,0}
\definecolor{gray}{rgb}{0.5,0.5,0.5}
\definecolor{mauve}{rgb}{0.58,0,0.82}
\lstset{
	numbers=left,  
	frame=tb,
	aboveskip=3mm,
	belowskip=3mm,
	showstringspaces=false,
	columns=flexible,
	framerule=1pt,
	rulecolor=\color{gray!35},
	backgroundcolor=\color{gray!5},
	basicstyle={\ttfamily},
	numberstyle=\tiny\color{gray},
	keywordstyle=\color{blue},
	commentstyle=\color{dkgreen},
	stringstyle=\color{mauve},
	breaklines=true,
	breakatwhitespace=true,
	tabsize=3
}

\usetikzlibrary{karnaugh}

\begin{document}
	\makereporttitle
	{Радиоэлектроника и лазерная техника (РЛ)} % Название факультета
	{Технология приборостроения (РЛ6)} % Название кафедры
	{Лабораторной работе №1} % Название работы (в дат. падеже)
	{Цифровые устройства и микропроцессоры} % Название курса (необязательный аргумент)
	{Исследование дребезка контактов на кнопке.} % Тема работы
	{} % Номер варианта (необязательный аргумент)
	{Филимонов~С.~В./РЛ6-61} % Номер группы/ФИО студента (если авторов несколько, их необходимо разделить запятой)
	{Семеренко~Д.~А.} % ФИО преподавателя
	
	\tableofcontents

	\chapter{Реализация шифратора для вывода знака на ССИ.}
	На~рисунке~\ref{img:BCD} пример семисегментного индикатора.
	
\includeimage
	{BCD} % Имя файла без расширения (файл должен быть расположен в директории inc/img/)
	{f} % Обтекание (без обтекания)
	{h} % Положение рисунка (см. figure из пакета float)
	{0.25\textwidth} % Ширина рисунка
	{Семисегментный индикатор} % Подпись рисунка 
	
	Кодировка

	\begin{center}
		\begin{tabular}{ |c||c|c|c|c||c|c|c|c|c|c|c| } 
			\hline
		     Символ & $x_0$ & $x_1$ & $x_2$ & $x_3$ &  a & b & c & d & e & f & g \\
		    \hline
			 0 & 0 & 0 & 0 & 0 & 1 & 1 & 1 & 1 & 1 & 1 & 0 \\ 
			 1 & 0 & 0 & 0 & 1 & 0 & 1 & 1 & 0 & 0 & 0 & 0 \\
			 2 & 0 & 0 & 1 & 0 & 1 & 1 & 0 & 1 & 1 & 0 & 1 \\
			 3 & 0 & 0 & 1 & 1 & 1 & 1 & 1 & 1 & 0 & 0 & 1 \\
			 4 & 0 & 1 & 0 & 0 & 0 & 1 & 1 & 0 & 0 & 1 & 1 \\
			 5 & 0 & 1 & 0 & 1 & 1 & 0 & 1 & 1 & 0 & 1 & 1 \\
			 6 & 0 & 1 & 1 & 0 & 1 & 0 & 1 & 1 & 1 & 1 & 1 \\
			 7 & 0 & 1 & 1 & 1 & 1 & 1 & 1 & 0 & 0 & 0 & 0 \\
			 8 & 1 & 0 & 0 & 0 & 1 & 1 & 1 & 1 & 1 & 1 & 1 \\
			 9 & 1 & 0 & 0 & 1 & 1 & 1 & 1 & 1 & 0 & 1 & 1 \\
			 Л & 1 & 0 & 1 & 0 & 1 & 1 & 1 & 0 & 1 & 1 & 0 \\
			\hline
		\end{tabular}
	\end{center}
	
	\section{Алгебраические уравнения в СКНФ и СДНФ}
	Определим СКНФ и СДНФ: \\
	$y^{\text{СКНФ}}_a = (x_0 \vee x_1 \vee x_2 \vee \bar{x_3}) \cdot
						 (x_0 \vee \bar{x_1} \vee x_2 \vee x_3) $ \\
	$y^{\text{СДНФ}}_a = \bar{x_0} \bar{x_1} \bar{x_2} \bar{x_3} \vee
						 \bar{x_0} \bar{x_1} x_2 \bar{x_3} \vee 
						 \bar{x_0} \bar{x_1} x_2 x_3 \vee
						 \bar{x_0} x_1 \bar{x_2} x_3 \vee 
					     \bar{x_0} x_1 x_2 \bar{x_3} \vee 
						 \bar{x_0} x_1 x_2 x_3 \vee 
	                     x_0 \bar{x_1} \bar{x_2} \bar{x_3} \vee 
	                     x_0 \bar{x_1} \bar{x_2} x_3 \vee 
	                     x_0 \bar{x_1} x_2 \bar{x_3}$
	\\
	\\             
	$y^{\text{СКНФ}}_b = (x_0 \vee \bar{x_1} \vee x_2 \vee \bar{x_3}) \cdot 
						 (x_0 \vee \bar{x_1} \vee \bar{x_2} \vee x_3)$ \\ 
	$y^{\text{СДНФ}}_b = \bar{x_0} \bar{x_1} \bar{x_2} \bar{x_3} \vee 
						 \bar{x_0} \bar{x_1} \bar{x_2} x_3 \vee
						 \bar{x_0} \bar{x_1} x_2 \bar{x_3} \vee 
						 \bar{x_0} \bar{x_1} x_2 x_3 \vee
						 \bar{x_0} x_1 \bar{x_2} \bar{x_3} \vee 
						 \bar{x_0} x_1 x_2 x_3 \vee 
						 x_0 \bar{x_1} \bar{x_2} \bar{x_3} \vee 
						 x_0 \bar{x_1} \bar{x_2} x_3 \vee 
						 x_0 \bar{x_1} x_2 \bar{x_3}$ 
	\\
	\\
	$y^{\text{СКНФ}}_c = (x_0 \vee x_1 \vee \bar{x_2} \vee x_3)$ \\           
    $y^{\text{СДНФ}}_c = \bar{x_0} \bar{x_1} \bar{x_2} \bar{x_3} \vee
    					 \bar{x_0} \bar{x_1} \bar{x_2} x_3 \vee 
    					 \bar{x_0} \bar{x_1} x_2 x_3 \vee
    					 \bar{x_0} x_1 \bar{x_2} \bar{x_3} \vee 
				     	 \bar{x_0} x_1 \bar{x_2} x_3 \vee 
    					 \bar{x_0} x_1 x_2 \bar{x_3} \vee 
    					 \bar{x_0} x_1 x_2 x_3 \vee 
    					 x_0 \bar{x_1} \bar{x_2} \bar{x_3} \vee 
    					 x_0 \bar{x_1} \bar{x_2} x_3 \vee 
        				 x_0 \bar{x_1} x_2 \bar{x_3}$
	\\
	\\
    $y^{\text{СКНФ}}_d = (x_0 \vee x_1 \vee x_2 \vee \bar{x_3}) \cdot
    					 (x_0 \vee \bar{x_1} \vee x_2 \vee x_3) \cdot 
    					 (x_0 \vee \bar{x_1} \vee \bar{x_2} \vee \bar{x_3}) \cdot 
    					 (\bar{x_0} \vee x_1  \vee \bar{x_2} \vee x_3)$ \\
    $y^{\text{СДНФ}}_d = \bar{x_0} \bar{x_1} \bar{x_2} \bar{x_3} \vee
    					 \bar{x_0} \bar{x_1} x_2 \bar{x_3} \vee 
   						 \bar{x_0} \bar{x_1} x_2 x_3 \vee
   						 \bar{x_0} x_1 \bar{x_2} x_3 \vee 
   						 \bar{x_0} x_1 x_2 \bar{x_3} \vee 
   						 x_0 \bar{x_1} \bar{x_2} \bar{x_3} \vee 
   						 x_0 \bar{x_1} \bar{x_2} x_3$
	\\
	\\
    $y^{\text{СКНФ}}_e = (x_0 \vee x_1 \vee x_2 \vee \bar{x_3}) \cdot
    					 (x_0 \vee x_1 \vee \bar{x_2} \vee \bar{x_3}) \cdot
    					 (x_0 \vee \bar{x_1} \vee x_2 \vee x_3) \cdot 
    					 (x_0 \vee \bar{x_1} \vee x_2 \vee \bar{x_3}) \cdot 
    					 (x_0 \vee \bar{x_1} \vee \bar{x_2} \vee \bar{x_3}) \cdot 
    					 (\bar{x_0} \vee x_1 \vee x_2 \vee \bar{x_3})$ \\
    $y^{\text{СДНФ}}_e = \bar{x_0} \bar{x_1} \bar{x_2} \bar{x_3} \vee
    					 \bar{x_0} \bar{x_1} x_2 \bar{x_3} \vee 
    					 \bar{x_0} x_1 x_2 \bar{x_3} \vee
    					 x_0 \bar{x_1} \bar{x_2} \bar{x_3} \vee
    					 x_0 \bar{x_1} x_2 \bar{x_3}$
	\\
	\\
    $y^{\text{СКНФ}}_f = (x_0 \vee x_1 \vee x_2 \vee \bar{x_3}) \cdot 
    					 (x_0 \vee x_1 \vee \bar{x_2} \vee x_3) \cdot 
    					 (x_0 \vee x_1 \vee \bar{x_2} \vee \bar{x_3}) \cdot 
    					 (x_0 \vee \bar{x_1} \vee \bar{x_2} \vee \bar{x_3})$ \\     
    $y^{\text{СДНФ}}_f = \bar{x_0} \bar{x_1} \bar{x_2} \bar{x_3} \vee
    					 \bar{x_0} x_1 \bar{x_2} \bar{x_3} \vee 
    					 \bar{x_0} x_1 \bar{x_2} x_3 \vee
    					 \bar{x_0} x_1 x_2 \bar{x_3} \vee 
    					 x_0 \bar{x_1} \bar{x_2} \bar{x_3} \vee 
    					 x_0 \bar{x_1} \bar{x_2} x_3 \vee 
    					 x_0 \bar{x_1} x_2 \bar{x_3}$
	\\
	\\
    $y^{\text{СКНФ}}_g = (x_0 \vee x_1 \vee x_2 \vee x_3) \cdot 
    					 (x_0 \vee x_1 \vee x_2 \vee \bar{x_3}) \cdot 
    					 (x_0 \vee \bar{x_1} \vee \bar{x_2} \vee \bar{x_3}) \cdot 
    					 (\bar{x_0} \vee x_1 \vee \bar{x_2} \vee x_3)$ \\					            
    $y^{\text{СДНФ}}_g = \bar{x_0} \bar{x_1} x_2 \bar{x_3} \vee
    					 \bar{x_0} \bar{x_1} x_2 x_3 \vee 
    					 \bar{x_0} x_1 \bar{x_2} \bar{x_3} \vee
    					 \bar{x_0} x_1 \bar{x_2} x_3 \vee 
   						 \bar{x_0} x_1 x_2 \bar{x_3} \vee 
   						 x_0 \bar{x_1} \bar{x_2} \bar{x_3} \vee 
   						 x_0 \bar{x_1} \bar{x_2} x_3$

%=======================================================================================================================
	\section{Минимизация с помощью Карт Карно.}


	\centering{
    	\begin{tikzpicture}[karnaugh,
    						American style, 
    						grp/.style n args={3}{#1,fill=#1!30,
    							minimum width=#2\kmunitlength,
    							minimum height=#3\kmunitlength,
    							rounded corners=0.2\kmunitlength,
    							fill opacity=0.6,
    							rectangle,draw}]
			\karnaughmap{4}{$a^{\text{СДНФ}}$}{{$x_2$}{$x_0$}{$x_3$}{$x_1$}}%
			{0001 1111 0101 1001}{
				\node[grp={blue}{1.9}{1.9}](n000) at (3.0,3.0) {};
				\node[grp={green}{1.9}{1.9}](n001) at (2.0,2.0) {};
				\node[grp={red}{0.9}{1.9}](n002) at (1.5,1.0) {};
				\node[grp={cyan}{0.9}{0.9}](n004) at (3.5,3.5) {};
				\node[grp={cyan}{0.9}{0.9}](n005) at (3.5,0.5) {};
			}
		\end{tikzpicture}
    	\begin{tikzpicture}[karnaugh, American style]
    		\karnaughmap{4}{$a^{\text{СКНФ}}$}{{$x_2$}{$x_0$}{$x_3$}{$x_1$}}%
    		{1011 1111 0111 1111}{}
    	\end{tikzpicture} \\
		$y^{\text{СДНФ}}_a = x_2 \bar{x_0} \vee \bar{x_1} \bar{x_3} \vee x_0 \bar{x_1} \bar{x_2} \vee \bar{x_0} x_1 x_3$ \\
		$y^{\text{СКНФ}}_a = (x_0 \vee x_1 \vee x_2 \vee \bar{x_3}) \cdot (x_0 \vee \bar{x_1} \vee x_2 \vee x_3)$ \\
	}

	\centering{
	\begin{tikzpicture}[karnaugh,
						American style, 
						grp/.style n args={3}{#1,fill=#1!30,
							minimum width=#2\kmunitlength,
							minimum height=#3\kmunitlength,
							rounded corners=0.2\kmunitlength,
							fill opacity=0.6,
							rectangle,draw}]
					\karnaughmap{4}{$b^{\text{СДНФ}}$}{{$x_2$}{$x_0$}{$x_3$}{$x_1$}}%
					{0001 1101 0101 0111}{
						\node[grp={blue}{1.9}{0.9}](n000) at (3.0,3.5) {};
						\node[grp={green}{1.9}{0.9}](n001) at (2.0,2.5) {};
						\node[grp={red}{0.9}{1.9}](n002) at (1.5,1.0) {};
						\node[grp={yellow}{1.9}{0.9}](n003) at (3.0,1.5) {};
						\node[grp={cyan}{0.9}{0.9}](n004) at (2.5,3.5) {};
						\node[grp={cyan}{0.9}{0.9}](n005) at (2.5,0.5) {};
			}
		\end{tikzpicture}
		\begin{tikzpicture}[karnaugh, American style]
			\karnaughmap{4}{$b^{\text{СКНФ}}$}{{$x_2$}{$x_0$}{$x_3$}{$x_1$}}%
			{1110 1111 1011 1111}{}
		\end{tikzpicture} \\
		$y^{\text{СДНФ}}_b = \bar{x_1} x_2 \bar{x_3} \vee \bar{x_0} x_2 x_3 \vee \bar{x_0} \bar{x_1} x_3 \vee \bar{x_0} \bar{x_2} \bar{x_3} \vee x_0 \bar{x_1} \bar{x_3}$ \\
		$y^{\text{СКНФ}}_b = (x_0 \vee \bar{x_1} \vee x_2 \vee \bar{x_3}) \cdot (x_0 \vee \bar{x_1} \vee \bar{x_2} \vee x_3)$ \\ 
	}

	\centering{
	\begin{tikzpicture}[karnaugh,
		American style, 
		grp/.style n args={3}{#1,fill=#1!30,
			minimum width=#2\kmunitlength,
			minimum height=#3\kmunitlength,
			rounded corners=0.2\kmunitlength,
			fill opacity=0.6,
			rectangle,draw}]
		\karnaughmap{4}{$c^{\text{СДНФ}}$}{{$x_2$}{$x_0$}{$x_3$}{$x_1$}}%
		{0001 1110 0101 1111}{
			\node[grp={blue}{1.9}{1.9}](n000) at (3.0,1.0) {};
			\node[grp={green}{0.9}{1.9}](n001) at (1.5,1.0) {};
			\node[grp={red}{0.9}{1.9}](n002) at (1.5,2.0) {};
			\node[grp={yellow}{1.9}{0.9}](n003) at (3.0,3.5) {};
			\node[grp={cyan}{0.9}{1.9}](n004) at (3.5,3.0) {};
		}
	\end{tikzpicture}
	\begin{tikzpicture}[karnaugh, American style]
		\karnaughmap{4}{$c^{\text{СКНФ}}$}{{$x_2$}{$x_0$}{$x_3$}{$x_1$}}%
		{1101 1111 1111 1111}{}
	\end{tikzpicture}
	\\
	$y^{\text{СДНФ}}_c = \bar{x_0} \bar{x_2} \vee x_0 \bar{x_1} \bar{x_2} \vee x_0 \bar{x_1} \bar{x_3} \vee \bar{x_0} x_2 x_3 \vee \bar{x_0} x_1 x_2$
	\\
	$y^{\text{СКНФ}}_c = (x_0 \vee x_1 \vee \bar{x_2} \vee x_3)$
	\\      
	}

	\centering{
	\begin{tikzpicture}[karnaugh,
		American style, 
		grp/.style n args={3}{#1,fill=#1!30,
			minimum width=#2\kmunitlength,
			minimum height=#3\kmunitlength,
			rounded corners=0.2\kmunitlength,
			fill opacity=0.6,
			rectangle,draw}]
		\karnaughmap{4}{$d^{\text{СДНФ}}$}{{$x_2$}{$x_0$}{$x_3$}{$x_1$}}%
		{0000 0111 0101 1001}{
			\node[grp={blue}{0.9}{1.9}](n000) at (2.5,2.0) {};
			\node[grp={green}{0.9}{1.9}](n001) at (2.5,3.0) {};
			\node[grp={red}{1.9}{0.9}](n002) at (3.0,2.5) {};
			\node[grp={yellow}{0.9}{1.9}](n003) at (1.5,1.0) {};
		}
	\end{tikzpicture}
	\begin{tikzpicture}[karnaugh, American style]
		\karnaughmap{4}{$d^{\text{СКНФ}}$}{{$x_2$}{$x_0$}{$x_3$}{$x_1$}}%
		{1011 1101 0110 1111}{}
	\end{tikzpicture}
	\\
	$y^{\text{СДНФ}}_d = x_0 \bar{x_1} \bar{x_2} \vee \bar{x_0} \bar{x_1} x_2 \vee \bar{x_0} x_2 \bar{x_3} \vee \bar{x_0} \bar{x_1} \bar{x_3} \vee \bar{x_0} x_1 \bar{x_2} x_3$ 
	\\
	$y^{\text{СКНФ}}_d = (x_0 \vee x_1 \vee x_2 \vee \bar{x_3}) \cdot (x_0 \vee \bar{x_1} \vee x_2 \vee x_3) \cdot (x_0 \vee \bar{x_1} \vee \bar{x_2} \vee \bar{x_3}) \cdot (\bar{x_0} \vee x_1  \vee \bar{x_2} \vee x_3)$
	\\
	}

	\centering{
	\begin{tikzpicture}[karnaugh,
		American style, 
		grp/.style n args={3}{#1,fill=#1!30,
			minimum width=#2\kmunitlength,
			minimum height=#3\kmunitlength,
			rounded corners=0.2\kmunitlength,
			fill opacity=0.6,
			rectangle,draw}]
		\karnaughmap{4}{$e^{\text{СДНФ}}$}{{$x_2$}{$x_0$}{$x_3$}{$x_1$}}%
		{0001 0011 0001 0001}{
			\node[grp={blue}{1.9}{1.9}](n000) at (2.0,2.0) {};
			\node[grp={green}{1.9}{0.9}](n001) at (3.0,2.5) {};
		}
	\end{tikzpicture}
	\begin{tikzpicture}[karnaugh,
		American style, 
		grp/.style n args={3}{#1,fill=#1!30,
			minimum width=#2\kmunitlength,
			minimum height=#3\kmunitlength,
			rounded corners=0.2\kmunitlength,
			fill opacity=0.6,
			rectangle,draw}]
		\karnaughmap{4}{$e^{\text{СКНФ}}$}{{$x_2$}{$x_0$}{$x_3$}{$x_1$}}%
		{1011 1111 0000 0111}{
			\node[grp={blue}{1.9}{1.9}](n000) at (1.0,1.0) {};
			\node[grp={green}{0.9}{0.9}](n001) at (1.5,3.5) {};
			\node[grp={green}{0.9}{0.9}](n001) at (1.5,0.5) {};
			\node[grp={red}{0.9}{0.9}](n002) at (0.5,0.5) {};
			\node[grp={red}{0.9}{0.9}](n002) at (3.5,0.5) {};
		}
	\end{tikzpicture}
	\\
	$y^{\text{СДНФ}}_e = \bar{x_1} \bar{x_3} \vee \bar{x_0} x_2 \bar{x_3}$ 
	\\
	$y^{\text{СКНФ}}_e = (x_0 \vee \bar{x_3}) \cdot (x_0 \vee \bar{x_1} \vee x_2) \cdot (x_1 \vee x_2 \vee \bar{x_3})$
	\\
	}
	\centering{
	\begin{tikzpicture}[karnaugh,
		American style, 
		grp/.style n args={3}{#1,fill=#1!30,
			minimum width=#2\kmunitlength,
			minimum height=#3\kmunitlength,
			rounded corners=0.2\kmunitlength,
			fill opacity=0.6,
			rectangle,draw}]
		\karnaughmap{4}{$f^{\text{СДНФ}}$}{{$x_2$}{$x_0$}{$x_3$}{$x_1$}}%
		{0001 0010 0101 1011}{
			\node[grp={blue}{0.9}{1.9}](n000) at (1.5,2.0) {};
			\node[grp={green}{0.9}{1.9}](n001) at (1.5,1.0) {};
			\node[grp={red}{1.9}{0.9}](n002) at (3.0,1.5) {};
			\node[grp={yellow}{0.9}{1.9}](n003) at (3.5,2.0) {};
			\node[grp={cyan}{0.9}{1.9}](n004) at (3.5,1.0) {};
		}
	\end{tikzpicture}
	\begin{tikzpicture}[karnaugh,
		American style, 
		grp/.style n args={3}{#1,fill=#1!30,
			minimum width=#2\kmunitlength,
			minimum height=#3\kmunitlength,
			rounded corners=0.2\kmunitlength,
			fill opacity=0.6,
			rectangle,draw}]
		\karnaughmap{4}{$f^{\text{СКНФ}}$}{{$x_2$}{$x_0$}{$x_3$}{$x_1$}}%
		{1101 1111 0100 1111}{
			\node[grp={blue}{0.9}{1.9}](n000) at (0.5,2.0) {};
			\node[grp={green}{0.9}{1.9}](n001) at (0.5,1.0) {};
			\node[grp={red}{1.9}{0.9}](n002) at (1.0,1.5) {};
		}
	\end{tikzpicture}
	\\
	$y^{\text{СДНФ}}_f = x_0 \bar{x_1} \bar{x_3} \vee x_0 \bar{x_1} \bar{x_2} \vee \bar{x_0} \bar{x_2} \bar{x_3} \vee \bar{x_0} x_1 x_3 \vee \bar{x_0} x_1 \bar{x_2}$ 
	\\
	$y^{\text{СКНФ}}_f = (x_0 \vee x_1 \vee \bar{x_2}) \cdot (x_0 \vee \bar{x_2} \vee \bar{x_3}) \cdot (x_0 \vee x_1 \vee \bar{x_3})$
	\\
	}


	\centering{
	\begin{tikzpicture}[karnaugh,
		American style, 
		grp/.style n args={3}{#1,fill=#1!30,
			minimum width=#2\kmunitlength,
			minimum height=#3\kmunitlength,
			rounded corners=0.2\kmunitlength,
			fill opacity=0.6,
			rectangle,draw}]
		\karnaughmap{4}{$g^{\text{СДНФ}}$}{{$x_2$}{$x_0$}{$x_3$}{$x_1$}}%
		{0001 0111 0001 1010}{
			\node[grp={blue}{0.9}{1.9}](n000) at (1.5,2.0) {};
			\node[grp={green}{0.9}{1.9}](n001) at (2.5,3.0) {};
			\node[grp={red}{0.9}{1.9}](n002) at (3.5,2.0) {};
			\node[grp={yellow}{0.9}{1.9}](n003) at (3.5,1.0) {};
		}
	\end{tikzpicture}
	\begin{tikzpicture}[karnaugh,
		American style, 
		grp/.style n args={3}{#1,fill=#1!30,
			minimum width=#2\kmunitlength,
			minimum height=#3\kmunitlength,
			rounded corners=0.2\kmunitlength,
			fill opacity=0.6,
			rectangle,draw}]
		\karnaughmap{4}{$g^{\text{СКНФ}}$}{{$x_2$}{$x_0$}{$x_3$}{$x_1$}}%
		{0111 1101 0110 1111}{
			\node[grp={blue}{0.9}{0.9}](n000) at (0.5,3.5) {};
			\node[grp={blue}{0.9}{0.9}](n000) at (0.5,0.5) {};
			%\node[grp={green}{0.9}{1.9}](n001) at (2.5,3.0) {};
			%\node[grp={red}{0.9}{1.9}](n002) at (3.5,2.0) {};
			%\node[grp={yellow}{0.9}{1.9}](n003) at (3.5,1.0) {};
		}
	\end{tikzpicture}
	\\
	$y^{\text{СДНФ}}_g = x_0 \bar{x_1} \bar{x_3} \vee \bar{x_0} \bar{x_1} x_2 \vee \bar{x_0} x_1 \bar{x_3} \vee \bar{x_0} x_1 \bar{x_2}$ 
	\\
	$y^{\text{СКНФ}}_g = (x_0 \vee x_1 \vee x_2) \cdot (x_0 \vee \bar{x_1} \vee \bar{x_2} \vee \bar{x_3}) \cdot (\bar{x_0} \vee x_1 \vee \bar{x_2} \vee x_3)$
	\\
	}

	\section{Перевод полученных выражений к базису 2И-НЕ/2ИЛИ-НЕ}
	При переводе в базис к изначальному алгебраическому уравнению применяется двойное отрицание, после чего используются законы де Морагана:
	
	\begin{center}
		$\overline{a \cdot b} = \bar{a} \vee \bar{b}$ \\
		$\overline{a \vee b} = \bar{a} \cdot \bar{b}$ \\
	\end{center}
	
	Чтобы не загромождать запись двойными отрицаниями, они будут опускаться после того, как будет показано их применение, то есть:
	
	\begin{center}
		$\overline{\overline{A \vee B \vee C \vee D}} =$
		$\overline{\bar{A} \cdot \bar{B} \cdot \bar{C} \cdot \bar{D}}$ \\
		$A \vee B \vee C \vee D =$
		$\overline{\overline{
				   \overline{\overline{A \vee B}} \vee
				   \overline{\overline{C \vee D}}
				}}$
	\end{center}

	\newpage

	\subsection{2И-НЕ}	
	$y^{\text{ДНФ}}_a = \overline{\overline{\bar{x_0} x_2 \vee \bar{x_1} \bar{x_3} \vee x_0 \bar{x_1} \bar{x_2} \vee \bar{x_0} x_1 x_2}} =$ \\
	$= \overline{\overline{\bar{x_0} x_2} \cdot \overline{\bar{x_0} \bar{x_3}} \cdot \overline{x_0 \bar{x_1} \bar{x_2}} \cdot \overline{\bar{x_0} x_1 x_2}}$ \\
	$y^{\text{КНФ}}_a = \overline{\overline{\overline{\overline{(x_0 \vee x_1 \vee x_2 \vee \bar{x_3})}} \cdot (x_0 \vee \bar{x_1} \vee x_2 \vee x_3)}} = $ \\
	$= \overline{\overline{(x_0 \vee x_1 \vee x_2 \vee \bar{x_3})}} \cdot \overline{\overline{(x_0 \vee \bar{x_1} \vee x_2 \vee x_3)}} = $ \\
	$= \overline{\bar{x_0} \bar{x_1} \bar{x_2} x_3} \cdot \overline{\bar{x_0} x_1 \bar{x_2} \bar{x_3}}$ \\
	
	$y^{\text{ДНФ}}_b = \overline{\overline{\bar{x_1} x_2 \bar{x_3} \vee \bar{x_0} x_2 x_3 \vee \bar{x_0} \bar{x_1} x_3 \vee \bar{x_0} \bar{x_2} \bar{x_3} \vee x_0 \bar{x_1} \bar{x_3}}} = $ \\
	$= \overline{\overline{\bar{x_1} x_2 \bar{x_3}} \cdot\overline{\bar{x_0} x_2 x_3} \cdot\overline{\bar{x_0} \bar{x_1} x_3} \cdot\overline{\bar{x_0} \bar{x_2} \bar{x_3}} \cdot \overline{x_0 \bar{x_1} \bar{x_3}}}$ \\
	$y^{\text{КНФ}}_b = \overline{\overline{\overline{\overline{(x_0 \vee \bar{x_1} \vee x_2 \bar{x_3})}} \cdot(x_0 \vee \bar{x_1} \vee \bar{x_2} \vee x_3) =}}$ \\
	$= \overline{\overline{(x_0 \vee \bar{x_1} \vee x_2 \bar{x_3})}} \cdot\overline{\overline{(x_0 \vee \bar{x_1} \vee \bar{x_2} \vee x_3)}} =$ \\
	$= \overline{\bar{x_0} x_1 \bar{x_2} x_3} \cdot \overline{\bar{x_0} x_1 x_2 \bar{x_3}}$ \\
	
	$y^{\text{ДНФ}}_c = \overline{\overline{\bar{x_0} \bar{x_2} \vee x_0 \bar{x_1} \bar{x_2} \vee x_0 \bar{x_1} \bar{x_3} \vee\bar{x_0} x_2 x_3 \vee\bar{x_0} x_1 x_2}} =$ \\
	$= \overline{\overline{\bar{x_0} \bar{x_2}} \cdot\overline{x_0 \bar{x_1} \bar{x_2}} \cdot\overline{x_0 \bar{x_1} \bar{x_3}} \cdot\overline{\bar{x_0} x_2 x_3} \cdot \overline{\bar{x_0} x_1 x_2}}$ \\
	$y^{\text{КНФ}}_c = \overline{\overline{(x_0 \vee x_1 \vee \bar{x_2} \vee x_3)}} =$ $ \overline{\bar{x_0} \bar{x_1} x_2 \bar{x_3}}$ \\
	
	$y^{\text{ДНФ}}_d = \overline{\overline{x_0 \bar{x_1} \bar{x_2} \vee\bar{x_0} \bar{x_1} x_2 \vee\bar{x_0} x_2 \bar{x_3} \vee\bar{x_0} \bar{x_1} \bar{x_3} \vee\bar{x_0} x_1 \bar{x_2} x_3}} =$ \\
	$\overline{\overline{x_0 \bar{x_1} \bar{x_2}} \cdot \overline{\bar{x_0} \bar{x_1} x_2} \cdot \overline{\bar{x_0} x_2 \bar{x_3}} \cdot \overline{\bar{x_0} \bar{x_1} \bar{x_3}} \cdot \overline{\bar{x_0} x_1 \bar{x_2} x_3}}$ \\
	$y^{\text{КНФ}}_d = \overline{\overline{\overline{\overline{(x_0 \vee x_1 \vee x_2 \vee \bar{x_3}) \cdot(x_0 \vee \bar{x_1} \vee x_2 \vee x_3) \cdot(x_0 \vee \bar{x_1} \vee \bar{x_2} \vee \bar{x_3})}} \cdot (\bar{x_0} \vee x_1 \vee \bar{x_2} \vee x_3)}}$ \\
	$= \overline{\overline{(x_0 \vee x_1 \vee x_2 \vee \bar{x_3})}} \cdot \overline{\overline{(x_0 \vee \bar{x_1} \vee x_2 \vee x_3)}} \cdot\overline{\overline{(x_0 \vee \bar{x_1} \vee \bar{x_2} \vee \bar{x_3})}} \cdot \overline{\overline{(\bar{x_0} \vee x_1 \vee \bar{x_2} \vee x_3)}}$ \\
	$= \overline{\bar{x_0} \bar{x_1} \bar{x_2} x_3} \cdot \overline{\bar{x_0} x_1 \bar{x_2} \bar{x_3}} \cdot \overline{\bar{x_0} x_1 x_2 x_3} \cdot \overline{x_0 \bar{x_1} x_2 \bar{x_3}}$ \\
	
	$y^{\text{ДНФ}}_e = \overline{\overline{\bar{x_1} \bar{x_3} \vee \bar{x_0} x_2 \bar{x_3}}} = \overline{\overline{\bar{x_1} \bar{x_3}} \cdot \overline{\bar{x_0} x_2 \bar{x_3}}}$ \\
	$y^{\text{КНФ}}_e = \overline{\overline{\overline{\overline{(x_0 \vee \bar{x_3}) \vee (x_0 \vee \bar{x_1} \vee x_2)}} \cdot (x_1 \vee x_2 \vee \bar{x_3})}} =$ \\
	$= \overline{\overline{(x_0 \vee \bar{x_3})}} \cdot \overline{\overline{(x_0 \vee \bar{x_1} \vee x_2)}} \cdot \overline{\overline{(x_1 \vee x_2 \vee \bar{x_3})}} =$ \\
	$= \overline{\bar{x_0} x_3} \cdot \overline{\bar{x_0} x_1 \bar{x_2}} \cdot \overline{\bar{x_1} \bar{x_2} x_3}$ \\
	
	$y^{\text{ДНФ}}_f = \overline{\overline{x_0 \bar{x_1} \bar{x_3} \vee x_0 \bar{x_1} \bar{x_2} \vee \bar{x_0} \bar{x_2} \bar{x_3} \vee \bar{x_0} x_1 x_2 \vee \bar{x_0} x_1 \bar{x_2}}} =$ \\
	$= \overline{\overline{x_0 \bar{x_1} \bar{x_3}} \cdot \overline{x_0 \bar{x_1} \bar{x_2}} \cdot \overline{\bar{x_0} \bar{x_2} \bar{x_3}} \cdot \overline{\bar{x_0} x_1 x_3} \cdot \overline{\bar{x_0} x_1 \bar{x_2}}}$ \\
	$y^{\text{КНФ}}_f = \overline{\overline{\overline{\overline{(x_0 \vee x_1 \vee \bar{x_2}) \cdot(x_0 \vee \bar{x_2} \vee \bar{x_3})}} \cdot (x_0 \vee x_1 \vee \bar{x_3})}} =$ \\
	$= \overline{\overline{(x_0 \vee x_1 \vee \bar{x_2})}} \cdot \overline{\overline{(x_0 \vee \bar{x_2} \vee \bar{x_3})}} \cdot \overline{\overline{(x_0 \vee x_1 \vee \bar{x_3})}}$
	$= \overline{\bar{x_0} \bar{x_1} x_2} \cdot \overline{\bar{x_0} x_2 x_3} \cdot \overline{\bar{x_0} \bar{x_1} x_3}$ \\
	
	$y^{\text{ДНФ}}_g = \overline{\overline{x_0 \bar{x_1} \bar{x_3} \vee \bar{x_0} \bar{x_1} x_2 \vee \bar{x_0} x_1 \bar{x_3} \vee \bar{x_0} x_1 \bar{x_2}}}$
	$= \overline{\overline{x_0 \bar{x_1} \bar{x_3}} \cdot \overline{\bar{x_0} \bar{x_1} x_2} \cdot \overline{\bar{x_0} x_1 \bar{x_3}} \cdot \overline{\bar{x_0} x_1 \bar{x_2}}}$ \\
	$y^{\text{КНФ}}_g = \overline{\overline{\overline{\overline{(x_0 \vee x_1 \vee x_2) \cdot (x_0 \vee \bar{x_1} \vee \bar{x_2} \vee \bar{x_3})}} \cdot (\bar{x_0} \vee x_1 \vee \bar{x_2} \vee x_3)}} =$ \\
	$= \overline{\overline{(x_0 \vee x_1 \vee x_2)}} \cdot \overline{\overline{(x_0 \vee \bar{x_1} \vee \bar{x_2} \vee \bar{x_3})}} \cdot \overline{\overline{(\bar{x_0} \vee x_1 \vee \bar{x_2} \vee x_3)}} =$ \\
	$= \overline{\bar{x_0} \bar{x_1} \bar{x_2}} \cdot \overline{\bar{x_0} x_1 x_2 x_3} \cdot \overline{x_0 \bar{x_1} x_2 \bar{x_3}}$ \\
	
	\section{Цифровая схема}
	\begin{flushleft}
		Все схемы строились через КНФ.
	\end{flushleft}
	
	\includeimage
	{a_circuit}
	{f} % Обтекание (без обтекания)
	{h} % Положение рисунка (см. figure из пакета float)
	{1.0\textwidth} % Ширина рисунка
	{Схема для светодиода А.} % Подпись рисунка
	\includeimage
	{a_waveform}
	{f} % Обтекание (без обтекания)
	{h} % Положение рисунка (см. figure из пакета float)
	{1.0\textwidth} % Ширина рисунка
	{Сигнал на светодиоде А.} % Подпись рисунка
	
	\includeimage
	{b_circuit}
	{f} % Обтекание (без обтекания)
	{H} % Положение рисунка (см. figure из пакета float)
	{1.0\textwidth} % Ширина рисунка
	{Схема для светодиода B.} % Подпись рисунка
	\includeimage
	{b_waveform}
	{f} % Обтекание (без обтекания)
	{H} % Положение рисунка (см. figure из пакета float)
	{1.0\textwidth} % Ширина рисунка
	{Сигнал на светодиоде B.} % Подпись рисунка
	
	\includeimage
	{c_circuit}
	{f} % Обтекание (без обтекания)
	{H} % Положение рисунка (см. figure из пакета float)
	{1.0\textwidth} % Ширина рисунка
	{Схема для светодиода C.} % Подпись рисунка
	\includeimage
	{c_waveform}
	{f} % Обтекание (без обтекания)
	{H} % Положение рисунка (см. figure из пакета float)
	{1.0\textwidth} % Ширина рисунка
	{Сигнал на светодиоде C.} % Подпись рисунка
	
	\includeimage
	{d_circuit}
	{f} % Обтекание (без обтекания)
	{H} % Положение рисунка (см. figure из пакета float)
	{1.0\textwidth} % Ширина рисунка
	{Схема для светодиода D.} % Подпись рисунка
	\includeimage
	{d_waveform}
	{f} % Обтекание (без обтекания)
	{H} % Положение рисунка (см. figure из пакета float)
	{1.0\textwidth} % Ширина рисунка
	{Сигнал на светодиоде D.} % Подпись рисунка
	
	\includeimage
	{e_circuit}
	{f} % Обтекание (без обтекания)
	{H} % Положение рисунка (см. figure из пакета float)
	{1.0\textwidth} % Ширина рисунка
	{Схема для светодиода E.} % Подпись рисунка
	\includeimage
	{e_waveform}
	{f} % Обтекание (без обтекания)
	{H} % Положение рисунка (см. figure из пакета float)
	{1.0\textwidth} % Ширина рисунка
	{Сигнал на светодиоде E.} % Подпись рисунка
	
	\includeimage
	{f_circuit}
	{f} % Обтекание (без обтекания)
	{H} % Положение рисунка (см. figure из пакета float)
	{1.0\textwidth} % Ширина рисунка
	{Схема для светодиода F.} % Подпись рисунка
	\includeimage
	{f_waveform}
	{f} % Обтекание (без обтекания)
	{H} % Положение рисунка (см. figure из пакета float)
	{1.0\textwidth} % Ширина рисунка
	{Сигнал на светодиоде F.} % Подпись рисунка
	
	\includeimage
	{g_circuit}
	{f} % Обтекание (без обтекания)
	{H} % Положение рисунка (см. figure из пакета float)
	{1.0\textwidth} % Ширина рисунка
	{Схема для светодиода G.} % Подпись рисунка
	\includeimage
	{g_waveform}
	{f} % Обтекание (без обтекания)
	{H} % Положение рисунка (см. figure из пакета float)
	{1.0\textwidth} % Ширина рисунка
	{Сигнал на светодиоде G.} % Подпись рисунка
	
	\includeimage
	{bcd_circuit}
	{f} % Обтекание (без обтекания)
	{H} % Положение рисунка (см. figure из пакета float)
	{1.0\textwidth} % Ширина рисунка
	{Схема для светодиода BSD.} % Подпись рисунка
	\includeimage
	{bcd_waveform}
	{f} % Обтекание (без обтекания)
	{H} % Положение рисунка (см. figure из пакета float)
	{1.0\textwidth} % Ширина рисунка
	{Сигнал на BSD.} % Подпись рисунка
	\includeimage
	{BCD_new}
	{f} % Обтекание (без обтекания)
	{H} % Положение рисунка (см. figure из пакета float)
	{1.0\textwidth} % Ширина рисунка
	{Схема для светодиода BSD(С отрицанием, так как в ПЛИС светодиоды на анодах).} % Подпись рисунка

	\newpage
		
	\section{Уточнение.}

	\subsection{Таймер}

	\begin{flushleft}
		Пропишем таймер, который будет считать 5 секунд.
		Так как на вход будет подан сигнал с частотой 50 МГц, то мы
		ничего не увидим, так как переключение бует слишком быстрым.
		Необходимо собрать делитель напряжения сигнала до частоты в
		50 Гц. Важно будет уточнить, что чтобы проверить схему в 
		Quartus II, сделаем сквозной выход [0], потому что в 
		противном случае сигнал будет нельзя проверить внутри 
		электронной системы так как он будет слишком растянут, в то 
		время перед тем как добавить его в ПЛИС надо будет переключить на [28].
	
	\end{flushleft}

\begin{lstlisting}[language=verilog,escapeinside=``]
module Timer(clk, reset, out_pos);
	input clk;
	input reset;
	output reg [28:0]out_pos;

	always @(posedge clk)
	begin
		if (reset == 0)
			out_pos = 0;
		else
			out_pos <= out_pos + 1'd1;
	end
endmodule
\end{lstlisting}

	\subsection{Переадресатор.}
	
	\begin{flushleft}	
		В ходе тестов всплыла ошибка. что сигнал подается на элементы не корректно,
		Ошибка была в неправельных сигнал поступающих на BCD,
		для решения ее использовался элемент, который перенаправляет выходы.
		С X0, X1, X2, X3 -> X3, X2, X1, X0.
	\end{flushleft}
	
	\includeimage
	{redirector_circuit}
	{f} % Обтекание (без обтекания)
	{H} % Положение рисунка (см. figure из пакета float)
	{1.0\textwidth} % Ширина рисунка
	{Схема переадресатора.} % Подпись рисунка
	
	\subsection{Мультиплекстор.}
	
	\includeimage
	{multiplexer_circuit}
	{f} % Обтекание (без обтекания)
	{H} % Положение рисунка (см. figure из пакета float)
	{1.0\textwidth} % Ширина рисунка
	{Схема мультиплекстора.} % Подпись рисунка
	
	\includeimage
	{multiplexer_waveform}
	{f} % Обтекание (без обтекания)
	{H} % Положение рисунка (см. figure из пакета float)
	{1.0\textwidth} % Ширина рисунка
	{Временные диаграммы мультиплекстора.} % Подпись рисунка

%=======================================================================================================================
	\chapter{Десятичный счетчик.}
	
	\begin{flushleft}
		Так как нам надо будет считать до 99 по заданию, то для этого необходимо собрать счетчик. 
		Возможны две реализации счетчика: (1) Бинарная и (2) Десятичная. Я решил реализовать дестичную версию,
		так как у нас только два числа. Но важным уточнением будет, что правильнее было бы собрать бинарный счетчик.
	\end{flushleft}

	\section{Cчетчики.}

	\begin{flushleft}
		Итак для начала необходимо определится с счечиком, он должен быть реверсиным,
		синхронным и срабатывать по заднему фронту, для того чтобы срабатывать при нажатии кнопки.
		Соберем две версии, с предустановкой и без предустановки.
	\end{flushleft}

	\subsection{Реверсивный счетчик.}

	\begin{flushleft}
		Простая схема синхронного счетчика с реверсивным ходом, по заднему фронту.
	\end{flushleft}

	\includeimage
	{reversff_without_circuit}
	{f} % Обтекание (без обтекания)
	{H} % Положение рисунка (см. figure из пакета float)
	{1.0\textwidth} % Ширина рисунка
	{Синхронный реверсивный счетчик по заднему фронту.} % Подпись рисунка
	
	\includeimage
	{reversff_without_waveform}
	{f} % Обтекание (без обтекания)
	{H} % Положение рисунка (см. figure из пакета float)
	{1.0\textwidth} % Ширина рисунка
	{Временный диаграммы.} % Подпись рисунка

	\subsection{Реверсивный счетчик с предустановкой.}

	\begin{flushleft}
		Схема синхронного счетчика с реверсивным ходом, по заднему фронту. 
		С возможностью предустановки значения. Важным уточнением является, 
		то что для того чтобы использовать предустановку надо подать сигнал на маркер.
	\end{flushleft}
	
	\includeimage
	{reversff_with_circuit}
	{f} % Обтекание (без обтекания)
	{H} % Положение рисунка (см. figure из пакета float)
	{1.0\textwidth} % Ширина рисунка
	{Синхронный реверсивный счетчик по заднему фронту с предустановкой.} % Подпись рисунка
	
	\includeimage
	{reversff_with_waveform}
	{f} % Обтекание (без обтекания)
	{H} % Положение рисунка (см. figure из пакета float)
	{1.0\textwidth} % Ширина рисунка
	{Временные диаграммы.} % Подпись рисунка

	\section{Маркеры числа.}

	\begin{flushleft}
		В дальнейшем нам понадобится возможность узнавать какой значение на шине, 
		для этого придется использовать один из трех маркеров(0, 9, 10). 
	\end{flushleft}

	\subsection{Маркер 0.}
	
	\includeimage
	{MRK_zero_circuit}
	{f} % Обтекание (без обтекания)
	{H} % Положение рисунка (см. figure из пакета float)
	{1.0\textwidth} % Ширина рисунка
	{Схема маркера 0.} % Подпись рисунка
	
	\includeimage
	{MRK_zero_waveform}
	{f} % Обтекание (без обтекания)
	{H} % Положение рисунка (см. figure из пакета float)
	{1.0\textwidth} % Ширина рисунка
	{Временные диаграммы.} % Подпись рисунка
	
	\subsection{Маркер 9.}	
	
	\includeimage
	{MRK_nine_circuit}
	{f} % Обтекание (без обтекания)
	{H} % Положение рисунка (см. figure из пакета float)
	{1.0\textwidth} % Ширина рисунка
	{Схема маркера 9.} % Подпись рисунка
	
	\includeimage
	{MRK_nine_waveform}
	{f} % Обтекание (без обтекания)
	{H} % Положение рисунка (см. figure из пакета float)
	{1.0\textwidth} % Ширина рисунка
	{Временные диаграммы.} % Подпись рисунка
	
	\subsection{Маркер 10.}

	\includeimage
	{MRK_ten_circuit}
	{f} % Обтекание (без обтекания)
	{H} % Положение рисунка (см. figure из пакета float)
	{1.0\textwidth} % Ширина рисунка
	{Схема маркера 10.} % Подпись рисунка
	
	\includeimage
	{MRK_ten_waveform}
	{f} % Обтекание (без обтекания)
	{H} % Положение рисунка (см. figure из пакета float)
	{1.0\textwidth} % Ширина рисунка
	{Временные диаграммы.} % Подпись рисунка

	\section{Десятичный счетчик, без защиты.}

	\begin{flushleft}
		Итак у нас есть все элементы для того чтобы собрать простой дестичный счетчик, 
		в котором не будет никаких защит, и маркеров символизирующих нам о том или ином значении.
	\end{flushleft}

	\includeimage
	{ten_CNT_circuit}
	{f} % Обтекание (без обтекания)
	{H} % Положение рисунка (см. figure из пакета float)
	{1.0\textwidth} % Ширина рисунка
	{Счетчик десятичный, без предохранителя.} % Подпись рисунка
	
	\includeimage
	{ten_CNT_waveform1}
	{f} % Обтекание (без обтекания)
	{H} % Положение рисунка (см. figure из пакета float)
	{1.0\textwidth} % Ширина рисунка
	{Временные диаграммы(1).} % Подпись рисунка
	
	\includeimage
	{ten_CNT_waveform2}
	{f} % Обтекание (без обтекания)
	{H} % Положение рисунка (см. figure из пакета float)
	{1.0\textwidth} % Ширина рисунка
	{Временные диаграммы(2).} % Подпись рисунка

	\begin{flushleft}
		P. s. Важно будет уточнить, что возможны ошибки,
		в дальнеешем при запуске платы
		из за того что для десяток единиц разные счетчики. 
		Так что возможное место ошибок может быть в них
	\end{flushleft}

	\section{Блокиратор.}

	\begin{flushleft}
		Для начала собрем блокиратор, так называемую "Защита от дурака". 
		Идея заключается в том, что при реверсивнымом ходе мы блокируем
		сигнал если он пришел к нулю и блокируем сигнал, если он без 
		включенного реверса пришел к 99. Для этого на вход подаем
		сигнал с 00 или 99.
	\end{flushleft}

	\includeimage
	{blocker_circuit}
	{f} % Обтекание (без обтекания)
	{H} % Положение рисунка (см. figure из пакета float)
	{1.0\textwidth} % Ширина рисунка
	{Схема блокиратора.} % Подпись рисунка
	
	\includeimage
	{blocker_waveform}
	{f} % Обтекание (без обтекания)
	{H} % Положение рисунка (см. figure из пакета float)
	{1.0\textwidth} % Ширина рисунка
	{Временные диаграммы.} % Подпись рисунка

	\section{Watch Dog.}

	\begin{flushleft}
		Теперь нам осталось собрать элемент, который оповещает нас
		о том равен ли сигнал на счетчике 00 или 99.
	\end{flushleft}

	\includeimage
	{watch_dog_circuit}
	{f} % Обтекание (без обтекания)
	{H} % Положение рисунка (см. figure из пакета float)
	{1.0\textwidth} % Ширина рисунка
	{Смотрящий за сигналом.} % Подпись рисунка
	
	\includeimage
	{watch_dog_waveform}
	{f} % Обтекание (без обтекания)
	{H} % Положение рисунка (см. figure из пакета float)
	{1.0\textwidth} % Ширина рисунка
	{Временные диаграммы.} % Подпись рисунка

	\section{Десятичный счетчик.}

	\begin{flushleft}	
		Время собрать рабочий счетчик, с защитой и статусными сигналами
	\end{flushleft}
	
	\includeimage
	{CNT_new_circuit}
	{f} % Обтекание (без обтекания)
	{H} % Положение рисунка (см. figure из пакета float)
	{1.0\textwidth} % Ширина рисунка
	{Готовый счетчик.} % Подпись рисунка
	
	\includeimage
	{CNT_new_waveform}
	{f} % Обтекание (без обтекания)
	{H} % Положение рисунка (см. figure из пакета float)
	{1.0\textwidth} % Ширина рисунка
	{Временные диаграммы.} % Подпись рисунка

	\section{Счетчик со встроенным мультиплекстором.}

	\begin{flushleft}
		Ввиду того что на выходе мы имеем две шины, то их необходимо между
		собой переключать, для этого нам понадобится большой мультиплекстор.
		Который мы сразу объеденим с счетчиком, чтобы в финальной
		схеме было меньше элементов.
	\end{flushleft}

	\includeimage
	{multiplexer_BIG_circuit}
	{f} % Обтекание (без обтекания)
	{H} % Положение рисунка (см. figure из пакета float)
	{1.0\textwidth} % Ширина рисунка
	{Мултиплекстор большой для шины счетчика десятков.} % Подпись рисунка
	
	\includeimage
	{multiplexer_BIG_waveform}
	{f} % Обтекание (без обтекания)
	{H} % Положение рисунка (см. figure из пакета float)
	{1.0\textwidth} % Ширина рисунка
	{Временные диаграммы.} % Подпись рисунка
	
	\includeimage
	{CNT_pp_MP}
	{f} % Обтекание (без обтекания)
	{H} % Положение рисунка (см. figure из пакета float)
	{1.0\textwidth} % Ширина рисунка
	{Схема десятичного счетчика совмещенного с БОЛЬШИМ мультиплекстором.} % Подпись рисунка

%=======================================================================================================================

	\chapter{Схема считывания сигнала с кнопки и механизм обратного счета.}
	
	\begin{flushleft}
		Необходимо реализовать механизм считывания сигнала с кнопки, 
		который будет запускать/сбрасывать сигнал с кнопки, 
		как только по времени прохидит 5 секунд начинать обратный счет.
	\end{flushleft}

	\section{RS-триггер.}
	
	\begin{flushleft}
		В этом нам понадобится RS-триггер, предлагаю немного изменить триггер из среды Quartus.
	\end{flushleft}

	\includeimage
	{my_rsff_circuit}
	{f} % Обтекание (без обтекания)
	{H} % Положение рисунка (см. figure из пакета float)
	{1.0\textwidth} % Ширина рисунка
	{Схема RS-триггера.} % Подпись рисунка
	
	\includeimage
	{my_rsff_waveform}
	{f} % Обтекание (без обтекания)
	{H} % Положение рисунка (см. figure из пакета float)
	{1.0\textwidth} % Ширина рисунка
	{Временные диаграммы.} % Подпись рисунка

	\section{Таймер.}
	
	\begin{flushleft}
		Таймер который начинает работу при нажатии на кнопку. 
		Каждый клик производит сброс счетчика, для сброса
		должно приходить внешнее значение сброса. В данный момент важно уточнить, 
		что в схеме обратный счет начинается спустя 1 секунду бездействия.
		Ввиду того что значение может менятся,
		и возможно такое что надо будет строить временные диаграммы,
		то берем с внешней шины таймера значение которые подаем на сброс.
	\end{flushleft}

	\includeimage
	{timer_with_click_circuit}
	{f} % Обтекание (без обтекания)
	{H} % Положение рисунка (см. figure из пакета float)
	{1.0\textwidth} % Ширина рисунка
	{Схема таймера с кликом.} % Подпись рисунка
	
	\includeimage
	{timer_with_click_waveform}
	{f} % Обтекание (без обтекания)
	{H} % Положение рисунка (см. figure из пакета float)
	{1.0\textwidth} % Ширина рисунка
	{Временные диаграммы.} % Подпись рисунка

	\section{Задержка.}

	\begin{flushleft}	
		Ввиду особенности принципа работы, сигнал который мы получаем с таймера его необходимо задержать.
	\end{flushleft}
	
	\includeimage
	{delay_circuit}
	{f} % Обтекание (без обтекания)
	{H} % Положение рисунка (см. figure из пакета float)
	{1.0\textwidth} % Ширина рисунка
	{Схема задержки.} % Подпись рисунка
	
	\includeimage
	{delay_waveform}
	{f} % Обтекание (без обтекания)
	{H} % Положение рисунка (см. figure из пакета float)
	{1.0\textwidth} % Ширина рисунка
	{Временные диаграммы.} % Подпись рисунка

	\section{Блок бесконечного счета.}
	
	\begin{flushleft}
		Блок с бесконечным счетом, схож примерно с таймером, но для удобства я сделал в нем еще сквозной канал с кнопкой.
	\end{flushleft}
	
	\includeimage
	{inf_cnt_circuit}
	{f} % Обтекание (без обтекания)
	{H} % Положение рисунка (см. figure из пакета float)
	{1.0\textwidth} % Ширина рисунка
	{Схема бесконечного счетчика.} % Подпись рисунка
	
	\includeimage
	{inf_cnt_waveform}
	{f} % Обтекание (без обтекания)
	{H} % Положение рисунка (см. figure из пакета float)
	{1.0\textwidth} % Ширина рисунка
	{Временные диаграммы.} % Подпись рисунка

	\section{Схема считывания сигнала с кнопки и механизм обратного счета.}

	\begin{flushleft}
		И так время собрать общую схему. На таймер приходит сигнал с кнопки, 
		запускает таймер, как только проходит 5 секунд (1 секунда, так как мы отлаживаем), 
		сигнал сбрасывает таймер, и переходит в задержку. С задержки он
		попадает на бесконечный счетчик и RS-триггер, который включает
		СТАБИЛЬНЫЙ реверсивный ход. Счетчик считает до 00.
	\end{flushleft}

	\includeimage
	{timer_pp_inf_cnt}
	{f} % Обтекание (без обтекания)
	{H} % Положение рисунка (см. figure из пакета float)
	{1.0\textwidth} % Ширина рисунка
	{Схема таймера с бесконечным счетчиком.} % Подпись рисунка

%=======================================================================================================================

	\chapter{Счетчик, для ислледования дребезга контактов.}

	\begin{flushleft}
		Все элементы готовы, настало время собрать воедино всю схему. Опускаю моменты связанные с условием.
	\end{flushleft}
	
	\section{Общая схема.}

	\includeimage
	{counter_circuit}
	{f} % Обтекание (без обтекания)
	{H} % Положение рисунка (см. figure из пакета float)
	{1.0\textwidth} % Ширина рисунка
	{Общая схема счетчика.} % Подпись рисунка
	
	\includeimage
	{counter_waveform}
	{f} % Обтекание (без обтекания)
	{H} % Положение рисунка (см. figure из пакета float)
	{1.0\textwidth} % Ширина рисунка
	{Временные диаграммы на низких задержках.} % Подпись рисунка

	\section{Результат.}

	\begin{flushleft}
		По неизвестным причинам качество снимков оказалось ужасным, 
		по этому я выложил видео с работой счетчика на Яндекс.Диск(https://disk.yandex.ru/d/J8X07rPAcTW99w).
	\end{flushleft}
	
	\includeimage
	{test1}
	{f} % Обтекание (без обтекания)
	{H} % Положение рисунка (см. figure из пакета float)
	{1.0\textwidth} % Ширина рисунка
	{Снимок счетчика 1.} % Подпись рисунка
	
	\includeimage
	{test2}
	{f} % Обтекание (без обтекания)
	{H} % Положение рисунка (см. figure из пакета float)
	{1.0\textwidth} % Ширина рисунка
	{Снимок счетчика 2.} % Подпись рисунка
	
	\includeimage
	{test3}
	{f} % Обтекание (без обтекания)
	{H} % Положение рисунка (см. figure из пакета float)
	{1.0\textwidth} % Ширина рисунка
	{Снимок счетчика 3.} % Подпись рисунка
	

\end{document}





























