\documentclass{bmstu}
\usepackage{multirow}
%\usepackage{karnaugh-map}
\usepackage{tikz}
\usepackage{float}
\usepackage{fancyhdr}
\usepackage{graphicx}
\usepackage{titlesec}
\usepackage{titletoc}
\usepackage{listings}
\usepackage{appendix}
\usepackage{bm, amsmath, amsfonts}
\usepackage{multirow}
\usepackage{hyperref}
\usepackage{subfig}
\usepackage{url}
\usepackage{multirow}
\usepackage{array} 
\usepackage{wrapfig} % in preamble

% set the code style
\RequirePackage{listings}
\RequirePackage{xcolor}
\definecolor{dkgreen}{rgb}{0,0.6,0}
\definecolor{gray}{rgb}{0.5,0.5,0.5}
\definecolor{mauve}{rgb}{0.58,0,0.82}
\lstset{
	numbers=left,  
	frame=tb,
	aboveskip=3mm,
	belowskip=3mm,
	showstringspaces=false,
	columns=flexible,
	framerule=1pt,
	rulecolor=\color{gray!35},
	backgroundcolor=\color{gray!5},
	basicstyle={\ttfamily},
	numberstyle=\tiny\color{gray},
	keywordstyle=\color{blue},
	commentstyle=\color{dkgreen},
	stringstyle=\color{mauve},
	breaklines=true,
	breakatwhitespace=true,
	tabsize=3
}

\usetikzlibrary{karnaugh}

\begin{document}
	\makereporttitle
	{Радиоэлектроника и лазерная техника (РЛ)} % Название факультета
	{Технология приборостроения (РЛ6)} % Название кафедры
	{домашней работе №1} % Название работы (в дат. падеже)
	{Цифровые устройства и микропроцессоры} % Название курса (необязательный аргумент)
	{ПЛИС Altera ССИ, минимизация алгебраических функций} % Тема работы
	{20Л274} % Номер варианта (необязательный аргумент)
	{Филимонов~С.~В./РЛ6-61} % Номер группы/ФИО студента (если авторов несколько, их необходимо разделить запятой)
	{Семеренко~Д.~А.} % ФИО преподавателя
	
	\tableofcontents

	\chapter{Реализация шифратора для вывода знака на ССИ.}
	На~рисунке~\ref{img:BCD} пример семисегментного индикатора.
	
\includeimage
	{BCD} % Имя файла без расширения (файл должен быть расположен в директории inc/img/)
	{f} % Обтекание (без обтекания)
	{h} % Положение рисунка (см. figure из пакета float)
	{0.25\textwidth} % Ширина рисунка
	{Семисегментный индикатор} % Подпись рисунка 
	
	Кодировка

	\begin{center}
		\begin{tabular}{ |c||c|c|c|c||c|c|c|c|c|c|c| } 
			\hline
		     Символ & $x_0$ & $x_1$ & $x_2$ & $x_3$ &  a & b & c & d & e & f & g \\
		    \hline
			 0 & 0 & 0 & 0 & 0 & 1 & 1 & 1 & 1 & 1 & 1 & 0 \\ 
			 1 & 0 & 0 & 0 & 1 & 0 & 1 & 1 & 0 & 0 & 0 & 0 \\
			 2 & 0 & 0 & 1 & 0 & 1 & 1 & 0 & 1 & 1 & 0 & 1 \\
			 3 & 0 & 0 & 1 & 1 & 1 & 1 & 1 & 1 & 0 & 0 & 1 \\
			 4 & 0 & 1 & 0 & 0 & 0 & 1 & 1 & 0 & 0 & 1 & 1 \\
			 5 & 0 & 1 & 0 & 1 & 1 & 0 & 1 & 1 & 0 & 1 & 1 \\
			 6 & 0 & 1 & 1 & 0 & 1 & 0 & 1 & 1 & 1 & 1 & 1 \\
			 7 & 0 & 1 & 1 & 1 & 1 & 1 & 1 & 0 & 0 & 0 & 0 \\
			 8 & 1 & 0 & 0 & 0 & 1 & 1 & 1 & 1 & 1 & 1 & 1 \\
			 9 & 1 & 0 & 0 & 1 & 1 & 1 & 1 & 1 & 0 & 1 & 1 \\
			 Л & 1 & 0 & 1 & 0 & 1 & 1 & 1 & 0 & 1 & 1 & 0 \\
			\hline
		\end{tabular}
	\end{center}
	
	\section{Алгебраические уравнения в СКНФ и СДНФ}
	Определим СКНФ и СДНФ: \\
	$y^{\text{СКНФ}}_a = (x_0 \vee x_1 \vee x_2 \vee \bar{x_3}) \cdot
						 (x_0 \vee \bar{x_1} \vee x_2 \vee x_3) $ \\
	$y^{\text{СДНФ}}_a = \bar{x_0} \bar{x_1} \bar{x_2} \bar{x_3} \vee
						 \bar{x_0} \bar{x_1} x_2 \bar{x_3} \vee 
						 \bar{x_0} \bar{x_1} x_2 x_3 \vee
						 \bar{x_0} x_1 \bar{x_2} x_3 \vee 
					     \bar{x_0} x_1 x_2 \bar{x_3} \vee 
						 \bar{x_0} x_1 x_2 x_3 \vee 
	                     x_0 \bar{x_1} \bar{x_2} \bar{x_3} \vee 
	                     x_0 \bar{x_1} \bar{x_2} x_3 \vee 
	                     x_0 \bar{x_1} x_2 \bar{x_3}$
	\\
	\\             
	$y^{\text{СКНФ}}_b = (x_0 \vee \bar{x_1} \vee x_2 \vee \bar{x_3}) \cdot 
						 (x_0 \vee \bar{x_1} \vee \bar{x_2} \vee x_3)$ \\ 
	$y^{\text{СДНФ}}_b = \bar{x_0} \bar{x_1} \bar{x_2} \bar{x_3} \vee 
						 \bar{x_0} \bar{x_1} \bar{x_2} x_3 \vee
						 \bar{x_0} \bar{x_1} x_2 \bar{x_3} \vee 
						 \bar{x_0} \bar{x_1} x_2 x_3 \vee
						 \bar{x_0} x_1 \bar{x_2} \bar{x_3} \vee 
						 \bar{x_0} x_1 x_2 x_3 \vee 
						 x_0 \bar{x_1} \bar{x_2} \bar{x_3} \vee 
						 x_0 \bar{x_1} \bar{x_2} x_3 \vee 
						 x_0 \bar{x_1} x_2 \bar{x_3}$ 
	\\
	\\
	$y^{\text{СКНФ}}_c = (x_0 \vee x_1 \vee \bar{x_2} \vee x_3)$ \\           
    $y^{\text{СДНФ}}_c = \bar{x_0} \bar{x_1} \bar{x_2} \bar{x_3} \vee
    					 \bar{x_0} \bar{x_1} \bar{x_2} x_3 \vee 
    					 \bar{x_0} \bar{x_1} x_2 x_3 \vee
    					 \bar{x_0} x_1 \bar{x_2} \bar{x_3} \vee 
				     	 \bar{x_0} x_1 \bar{x_2} x_3 \vee 
    					 \bar{x_0} x_1 x_2 \bar{x_3} \vee 
    					 \bar{x_0} x_1 x_2 x_3 \vee 
    					 x_0 \bar{x_1} \bar{x_2} \bar{x_3} \vee 
    					 x_0 \bar{x_1} \bar{x_2} x_3 \vee 
        				 x_0 \bar{x_1} x_2 \bar{x_3}$
	\\
	\\
    $y^{\text{СКНФ}}_d = (x_0 \vee x_1 \vee x_2 \vee \bar{x_3}) \cdot
    					 (x_0 \vee \bar{x_1} \vee x_2 \vee x_3) \cdot 
    					 (x_0 \vee \bar{x_1} \vee \bar{x_2} \vee \bar{x_3}) \cdot 
    					 (\bar{x_0} \vee x_1  \vee \bar{x_2} \vee x_3)$ \\
    $y^{\text{СДНФ}}_d = \bar{x_0} \bar{x_1} \bar{x_2} \bar{x_3} \vee
    					 \bar{x_0} \bar{x_1} x_2 \bar{x_3} \vee 
   						 \bar{x_0} \bar{x_1} x_2 x_3 \vee
   						 \bar{x_0} x_1 \bar{x_2} x_3 \vee 
   						 \bar{x_0} x_1 x_2 \bar{x_3} \vee 
   						 x_0 \bar{x_1} \bar{x_2} \bar{x_3} \vee 
   						 x_0 \bar{x_1} \bar{x_2} x_3$
	\\
	\\
    $y^{\text{СКНФ}}_e = (x_0 \vee x_1 \vee x_2 \vee \bar{x_3}) \cdot
    					 (x_0 \vee x_1 \vee \bar{x_2} \vee \bar{x_3}) \cdot
    					 (x_0 \vee \bar{x_1} \vee x_2 \vee x_3) \cdot 
    					 (x_0 \vee \bar{x_1} \vee x_2 \vee \bar{x_3}) \cdot 
    					 (x_0 \vee \bar{x_1} \vee \bar{x_2} \vee \bar{x_3}) \cdot 
    					 (\bar{x_0} \vee x_1 \vee x_2 \vee \bar{x_3})$ \\
    $y^{\text{СДНФ}}_e = \bar{x_0} \bar{x_1} \bar{x_2} \bar{x_3} \vee
    					 \bar{x_0} \bar{x_1} x_2 \bar{x_3} \vee 
    					 \bar{x_0} x_1 x_2 \bar{x_3} \vee
    					 x_0 \bar{x_1} \bar{x_2} \bar{x_3} \vee
    					 x_0 \bar{x_1} x_2 \bar{x_3}$
	\\
	\\
    $y^{\text{СКНФ}}_f = (x_0 \vee x_1 \vee x_2 \vee \bar{x_3}) \cdot 
    					 (x_0 \vee x_1 \vee \bar{x_2} \vee x_3) \cdot 
    					 (x_0 \vee x_1 \vee \bar{x_2} \vee \bar{x_3}) \cdot 
    					 (x_0 \vee \bar{x_1} \vee \bar{x_2} \vee \bar{x_3})$ \\     
    $y^{\text{СДНФ}}_f = \bar{x_0} \bar{x_1} \bar{x_2} \bar{x_3} \vee
    					 \bar{x_0} x_1 \bar{x_2} \bar{x_3} \vee 
    					 \bar{x_0} x_1 \bar{x_2} x_3 \vee
    					 \bar{x_0} x_1 x_2 \bar{x_3} \vee 
    					 x_0 \bar{x_1} \bar{x_2} \bar{x_3} \vee 
    					 x_0 \bar{x_1} \bar{x_2} x_3 \vee 
    					 x_0 \bar{x_1} x_2 \bar{x_3}$
	\\
	\\
    $y^{\text{СКНФ}}_g = (x_0 \vee x_1 \vee x_2 \vee x_3) \cdot 
    					 (x_0 \vee x_1 \vee x_2 \vee \bar{x_3}) \cdot 
    					 (x_0 \vee \bar{x_1} \vee \bar{x_2} \vee \bar{x_3}) \cdot 
    					 (\bar{x_0} \vee x_1 \vee \bar{x_2} \vee x_3)$ \\					            
    $y^{\text{СДНФ}}_g = \bar{x_0} \bar{x_1} x_2 \bar{x_3} \vee
    					 \bar{x_0} \bar{x_1} x_2 x_3 \vee 
    					 \bar{x_0} x_1 \bar{x_2} \bar{x_3} \vee
    					 \bar{x_0} x_1 \bar{x_2} x_3 \vee 
   						 \bar{x_0} x_1 x_2 \bar{x_3} \vee 
   						 x_0 \bar{x_1} \bar{x_2} \bar{x_3} \vee 
   						 x_0 \bar{x_1} \bar{x_2} x_3$

%=======================================================================================================================
	\section{Минимизация с помощью различных алгоритмов}
	\subsection{Законы алгебры логики}
	Для начала сократим выражение с  помощью законов алгебры логики:
	\\

	\begin{center}
	$y^{\text{СКНФ}}_a = (x_0 \vee x_1 \vee x_2 \vee \bar{x_3}) \cdot (x_0 \vee \bar{x_1} \vee x_2 \vee x_3) =  $ \\
	$ ((x_0 \vee x_2) \vee (x_1 \vee \bar{x_3})) \cdot ((x_0 \vee x_2) \vee (\bar{x_1} \vee x_3)) =  $ \\
	$ (x_0 \vee x_2) \vee ((x_1 \vee \bar{x_3}) \cdot (\bar{x_1} \vee x_3))$
	\\
	$y^{\text{СДНФ}}_a = \bar{x_0} \bar{x_1} \bar{x_2} \bar{x_3} \vee
						 \underline{\bar{x_0} \bar{x_1} x_2 \bar{x_3}} \vee 
					     \underline{\underline{\bar{x_0} \bar{x_1} x_2 x_3}} \vee
						 \bar{x_0} x_1 \bar{x_2} x_3 \vee 
						 \bar{x_0} x_1 x_2 \bar{x_3} \vee 
						 \underline{\underline{\bar{x_0} x_1 x_2 x_3}} \vee 
						 \underline{\underline{\underline{x_0 \bar{x_1} \bar{x_2} \bar{x_3}}}} \vee 
						\underline{\underline{\underline{ x_0 \bar{x_1} \bar{x_2} x_3}}} \vee 
						 \underline{x_0 \bar{x_1} x_2 \bar{x_3}} = $
	$\bar{x_0} \bar{x_1} \bar{x_2} \bar{x_3} \vee
	 \bar{x_0} x_2 x_3 \vee
 	 \bar{x_0} x_1 \bar{x_2} x_3 \vee 
  	 \bar{x_0} x_1 x_2 \bar{x_3} \vee 
     x_0 \bar{x_1} \bar{x_2} \vee 
 	 \bar{x_1} x_2 \bar{x_3}$
	\end{center}

	
	Используем закон дистрибутивности
	\begin{center}
	   \begin{tabular}{ |c| } 
	     \hline
	       $x \vee (y \cdot z) = (x \vee y) \cdot (x \vee z)$ \\
	     \hline
	     \end{tabular}
	\end{center}
    
	\begin{center}
		$y^{\text{СКНФ}}_b = (x_0 \vee \bar{x_1} \vee x_2 \vee \bar{x_3}) \cdot 
							 (x_0 \vee \bar{x_1} \vee \bar{x_2} \vee x_3) = $ \\
		$\bar{x_1} \vee x_0 \vee (x_2 \vee \bar{x_3} \cdot 
							 (\bar{x_2} \vee x_3))$
		\\ 
		$y^{\text{СДНФ}}_b = \underline{\bar{x_0} \bar{x_1} \bar{x_2} \bar{x_3}} \vee 
							 \underline{\bar{x_0} \bar{x_1} \bar{x_2} x_3} \vee
							 \underline{\underline{\bar{x_0} \bar{x_1} x_2 \bar{x_3}}} \vee 
							 \underline{\underline{\bar{x_0} \bar{x_1} x_2 x_3}} \vee
							 \bar{x_0} x_1 \bar{x_2} \bar{x_3} \vee 
							 \bar{x_0} x_1 x_2 x_3 \vee 
							 \underline{\underline{\underline{x_0 \bar{x_1} \bar{x_2} \bar{x_3}}}} \vee 
							 x_0 \bar{x_1} \bar{x_2} x_3 \vee 
							 \underline{\underline{\underline{x_0 \bar{x_1} x_2 \bar{x_3}}}} = $ \\
		$\underline{\bar{x_0} \bar{x_1} \bar{x_2}} \vee 
		 \underline{\bar{x_0} \bar{x_1} x_2} \vee
		 \bar{x_0} x_1 \bar{x_2} \bar{x_3} \vee 
		 \bar{x_0} x_1 x_2 x_3 \vee 
		 x_0 \bar{x_1} \bar{x_2} x_3 \vee
		 x_0 \bar{x_1} \bar{x_3} = $ \\
		$\bar{x_0} \bar{x_1} \vee
		 \bar{x_0} x_1 \bar{x_2} \bar{x_3} \vee 
		 \bar{x_0} x_1 x_2 x_3 \vee 
		 x_0 \bar{x_1} \bar{x_2} x_3 \vee
		 x_0 \bar{x_1} \bar{x_3} $ 		
	\end{center} 


	Для $y^{\text{СКНФ}}_c$ сокращать нечего $y^{\text{СКНФ}}_c = (x_0 \vee x_1 \vee \bar{x_2} \vee x_3)$.
	\begin{center}   
		$y^{\text{СДНФ}}_c = \underline{\bar{x_0} \bar{x_1} \bar{x_2} \bar{x_3}} \vee
							 \underline{\bar{x_0} \bar{x_1} \bar{x_2} x_3} \vee 
							 \bar{x_0} \bar{x_1} x_2 x_3 \vee
							 \underline{\underline{\bar{x_0} x_1 \bar{x_2} \bar{x_3}}} \vee 
							 \underline{\underline{\bar{x_0} x_1 \bar{x_2} x_3}} \vee 
							 \underline{\underline{\underline{\bar{x_0} x_1 x_2 \bar{x_3}}}} \vee 
							 \underline{\underline{\underline{\bar{x_0} x_1 x_2 x_3}}} \vee 
							 \underline{\underline{\underline{\underline{x_0 \bar{x_1} \bar{x_2} \bar{x_3}}}}} \vee 
							 \underline{\underline{\underline{\underline{x_0 \bar{x_1} \bar{x_2} x_3}}}} \vee 
							 x_0 \bar{x_1} x_2 \bar{x_3} = $ \\
		$\underline{\bar{x_0} \bar{x_1} \bar{x_2}} \vee
		 \bar{x_0} \bar{x_1} x_2 x_3 \vee
		 \underline{\underline{\bar{x_0} x_1 \bar{x_2}}} \vee
		 \underline{\underline{\bar{x_0} x_1 x_2}} \vee
		 \underline{x_0 \bar{x_1} \bar{x_2}} \vee
		 x_0 \bar{x_1} x_2 \bar{x_3} = $ \\
		$\bar{x_1} \bar{x_2} \vee
		 \bar{x_0} \bar{x_1} x_2 x_3 \vee
		 \bar{x_0} x_1 \vee
		 x_0 \bar{x_1} x_2 \bar{x_3}$
	\end{center}       


	\begin{center}
       $y^{\text{СКНФ}}_d = (x_0 \vee x_1 \vee x_2 \vee \bar{x_3}) \cdot
							(x_0 \vee \bar{x_1} \vee x_2 \vee x_3) \cdot 
							(x_0 \vee \bar{x_1} \vee \bar{x_2} \vee \bar{x_3}) \cdot 
							(\bar{x_0} \vee x_1  \vee \bar{x_2} \vee x_3) = $ \\
	   $(x_0 \vee (x_1 \vee x_2 \vee \bar{x_3}) \cdot 
	   			  (\bar{x_1} \vee x_2 \vee x_3) \cdot 
	   			  (\bar{x_1} \vee \bar{x_2} \vee \bar{x_3})
	    ) \cdot (\bar{x_0} \vee x_1  \vee \bar{x_2} \vee x_3)$ \\   
	   $y^{\text{СДНФ}}_d = \underline{\bar{x_0} \bar{x_1} \bar{x_2} \bar{x_3}} \vee
							\underline{\bar{x_0} \bar{x_1} x_2 \bar{x_3}} \vee 
							\bar{x_0} \bar{x_1} x_2 x_3 \vee
							\bar{x_0} x_1 \bar{x_2} x_3 \vee 
							\bar{x_0} x_1 x_2 \bar{x_3} \vee$ 
							$\underline{\underline{x_0 \bar{x_1} \bar{x_2} \bar{x_3}}} \vee 
							\underline{\underline{x_0 \bar{x_1} \bar{x_2} x_3}} = $ \\
	   $\bar{x_0} \bar{x_1} \bar{x_3} \vee
	    \bar{x_0} \bar{x_1} x_2 x_3 \vee
	    \bar{x_0} x_1 \bar{x_2} x_3 \vee 
	    \bar{x_0} x_1 x_2 \bar{x_3} \vee 
	    x_0 \bar{x_1} \bar{x_2}$				
	\end{center}


	\begin{center}
		$y^{\text{СКНФ}}_e = \underline{(x_0 \vee x_1 \vee x_2 \vee \bar{x_3})} \cdot
							 \underline{(x_0 \vee x_1 \vee \bar{x_2} \vee \bar{x_3})} \cdot
							 \underline{\underline{(x_0 \vee \bar{x_1} \vee x_2 \vee x_3)}} \cdot 
							 \underline{\underline{(x_0 \vee \bar{x_1} \vee x_2 \vee \bar{x_3})}} \cdot 
							 (x_0 \vee \bar{x_1} \vee \bar{x_2} \vee \bar{x_3}) \cdot 
							 (\bar{x_0} \vee x_1 \vee x_2 \vee \bar{x_3}) = $ \\
		$(x_0 \vee x_1 \vee \bar{x_3}) \cdot 
		 (x_0 \vee \bar{x_1} \vee x_2) \cdot
		 ((x_0 \vee \bar{x_1} \vee \bar{x_2}) \cdot (\bar{x_0} \vee x_1 \vee x_2)) \vee \bar{x_3}$ \\		             
		$y^{\text{СДНФ}}_e = \underline{\bar{x_0} \bar{x_1} \bar{x_2} \bar{x_3}} \vee
							 \underline{\bar{x_0} \bar{x_1} x_2 \bar{x_3}} \vee 
						     \bar{x_0} x_1 x_2 \bar{x_3} \vee
							 \underline{\underline{x_0 \bar{x_1} \bar{x_2} \bar{x_3}}} \vee
							 \underline{\underline{x_0 \bar{x_1} x_2 \bar{x_3}}} = $ \\
	   $\underline{\bar{x_0} \bar{x_1} \bar{x_3}} \vee
	    \bar{x_0} x_1 x_2 \bar{x_3} \vee
	    \underline{x_0 \bar{x_1} \bar{x_3}} = $ \\
	   $\bar{x_1} \bar{x_3} \vee
	    \bar{x_0} x_1 x_2 \bar{x_3}$
	\end{center}


	\begin{center}
	   $y^{\text{СКНФ}}_f = (x_0 \vee x_1 \vee x_2 \vee \bar{x_3}) \cdot 
							(x_0 \vee x_1 \vee \bar{x_2} \vee x_3) \cdot 
							(x_0 \vee x_1 \vee \bar{x_2} \vee \bar{x_3}) \cdot 
							(x_0 \vee \bar{x_1} \vee \bar{x_2} \vee \bar{x_3}) = $ \\
	   $(x_0 \vee x_1 \vee (x_2 \vee \bar{x_3}) \cdot (\bar{x_2} \vee x_3)) \cdot
	    (x_0 \vee \bar{x_2} \bar{x_3})$ \\
	   $y^{\text{СДНФ}}_f = \underline{\bar{x_0} \bar{x_1} \bar{x_2} \bar{x_3}} \vee
							\underline{\bar{x_0} x_1 \bar{x_2} \bar{x_3}} \vee 
							\bar{x_0} x_1 \bar{x_2} x_3 \vee
							\bar{x_0} x_1 x_2 \bar{x_3} \vee 
							\underline{\underline{x_0 \bar{x_1} \bar{x_2} \bar{x_3}}} \vee $
							$\underline{\underline{x_0 \bar{x_1} \bar{x_2} x_3}} \vee 
							x_0 \bar{x_1} x_2 \bar{x_3} = $ \\
	   $\bar{x_0} \bar{x_2} \bar{x_3} \vee 
	    \bar{x_0} x_1 \bar{x_2} x_3 \vee
	    \bar{x_0} x_1 x_2 \bar{x_3} \vee
	    x_0 \bar{x_1} \bar{x_2} \vee
	    x_0 \bar{x_1} x_2 \bar{x_3} $
	\end{center}


	\begin{center}
       $y^{\text{СКНФ}}_g = (x_0 \vee x_1 \vee x_2 \vee x_3) \cdot 
							(x_0 \vee x_1 \vee x_2 \vee \bar{x_3}) \cdot 
							(x_0 \vee \bar{x_1} \vee \bar{x_2} \vee \bar{x_3}) \cdot 
							(\bar{x_0} \vee x_1 \vee \bar{x_2} \vee x_3) = $ \\
	   $(x_0 \vee x_1 \vee x_2) \cdot
	    ((x_0 \vee \bar{x_1} \vee \bar{x_3}) \cdot (\bar{x_0} \vee x_1 \vee x_3) \vee \bar{x_2})$ \\		            
	   $y^{\text{СДНФ}}_g = \underline{\bar{x_0} \bar{x_1} x_2 \bar{x_3}} \vee
							\underline{\bar{x_0} \bar{x_1} x_2 x_3} \vee 
							\underline{\underline{\bar{x_0} x_1 \bar{x_2} \bar{x_3}}} \vee
							\underline{\underline{\bar{x_0} x_1 \bar{x_2} x_3}} \vee 
							\bar{x_0} x_1 x_2 \bar{x_3} \vee 
							\underline{\underline{\underline{x_0 \bar{x_1} \bar{x_2} \bar{x_3}}}} \vee$ 
							$\underline{\underline{\underline{x_0 \bar{x_1} \bar{x_2} x_3}}} = $ \\
	   $\bar{x_0} \bar{x_1} x_2 \vee
	    \bar{x_0} x_1 \bar{x_2} \vee
	    \bar{x_0} x_1 x_2 \bar{x_3} \vee
	    x_0 \bar{x_1} \bar{x_2}$
	\end{center}


%=======================================================================================================================
	\subsection{Карты Карно}
	\centering{
    	\begin{tikzpicture}[karnaugh,
    						American style, 
    						grp/.style n args={3}{#1,fill=#1!30,
    							minimum width=#2\kmunitlength,
    							minimum height=#3\kmunitlength,
    							rounded corners=0.2\kmunitlength,
    							fill opacity=0.6,
    							rectangle,draw}]
			\karnaughmap{4}{$a^{\text{СДНФ}}$}{{$x_2$}{$x_0$}{$x_3$}{$x_1$}}%
			{0001 1111 0101 1001}{
				\node[grp={blue}{1.9}{1.9}](n000) at (3.0,3.0) {};
				\node[grp={green}{1.9}{1.9}](n001) at (2.0,2.0) {};
				\node[grp={red}{0.9}{1.9}](n002) at (1.5,1.0) {};
				\node[grp={cyan}{0.9}{0.9}](n004) at (3.5,3.5) {};
				\node[grp={cyan}{0.9}{0.9}](n005) at (3.5,0.5) {};
			}
		\end{tikzpicture}
    	\begin{tikzpicture}[karnaugh, American style]
    		\karnaughmap{4}{$a^{\text{СКНФ}}$}{{$x_2$}{$x_0$}{$x_3$}{$x_1$}}%
    		{1011 1111 0111 1111}{}
    	\end{tikzpicture} \\
		$y^{\text{СДНФ}}_a = x_2 \bar{x_0} \vee \bar{x_1} \bar{x_3} \vee x_0 \bar{x_1} \bar{x_2} \vee \bar{x_0} x_1 x_3$ \\
		$y^{\text{СКНФ}}_a = (x_0 \vee x_1 \vee x_2 \vee \bar{x_3}) \cdot (x_0 \vee \bar{x_1} \vee x_2 \vee x_3)$ \\
	}

	\centering{
	\begin{tikzpicture}[karnaugh,
						American style, 
						grp/.style n args={3}{#1,fill=#1!30,
							minimum width=#2\kmunitlength,
							minimum height=#3\kmunitlength,
							rounded corners=0.2\kmunitlength,
							fill opacity=0.6,
							rectangle,draw}]
					\karnaughmap{4}{$b^{\text{СДНФ}}$}{{$x_2$}{$x_0$}{$x_3$}{$x_1$}}%
					{0001 1101 0101 0111}{
						\node[grp={blue}{1.9}{0.9}](n000) at (3.0,3.5) {};
						\node[grp={green}{1.9}{0.9}](n001) at (2.0,2.5) {};
						\node[grp={red}{0.9}{1.9}](n002) at (1.5,1.0) {};
						\node[grp={yellow}{1.9}{0.9}](n003) at (3.0,1.5) {};
						\node[grp={cyan}{0.9}{0.9}](n004) at (2.5,3.5) {};
						\node[grp={cyan}{0.9}{0.9}](n005) at (2.5,0.5) {};
			}
		\end{tikzpicture}
		\begin{tikzpicture}[karnaugh, American style]
			\karnaughmap{4}{$b^{\text{СКНФ}}$}{{$x_2$}{$x_0$}{$x_3$}{$x_1$}}%
			{1110 1111 1011 1111}{}
		\end{tikzpicture} \\
		$y^{\text{СДНФ}}_b = \bar{x_1} x_2 \bar{x_3} \vee \bar{x_0} x_2 x_3 \vee \bar{x_0} \bar{x_1} x_3 \vee \bar{x_0} \bar{x_2} \bar{x_3} \vee x_0 \bar{x_1} \bar{x_3}$ \\
		$y^{\text{СКНФ}}_b = (x_0 \vee \bar{x_1} \vee x_2 \vee \bar{x_3}) \cdot (x_0 \vee \bar{x_1} \vee \bar{x_2} \vee x_3)$ \\ 
	}

	\centering{
	\begin{tikzpicture}[karnaugh,
		American style, 
		grp/.style n args={3}{#1,fill=#1!30,
			minimum width=#2\kmunitlength,
			minimum height=#3\kmunitlength,
			rounded corners=0.2\kmunitlength,
			fill opacity=0.6,
			rectangle,draw}]
		\karnaughmap{4}{$c^{\text{СДНФ}}$}{{$x_2$}{$x_0$}{$x_3$}{$x_1$}}%
		{0001 1110 0101 1111}{
			\node[grp={blue}{1.9}{1.9}](n000) at (3.0,1.0) {};
			\node[grp={green}{0.9}{1.9}](n001) at (1.5,1.0) {};
			\node[grp={red}{0.9}{1.9}](n002) at (1.5,2.0) {};
			\node[grp={yellow}{1.9}{0.9}](n003) at (3.0,3.5) {};
			\node[grp={cyan}{0.9}{1.9}](n004) at (3.5,3.0) {};
		}
	\end{tikzpicture}
	\begin{tikzpicture}[karnaugh, American style]
		\karnaughmap{4}{$c^{\text{СКНФ}}$}{{$x_2$}{$x_0$}{$x_3$}{$x_1$}}%
		{1101 1111 1111 1111}{}
	\end{tikzpicture}
	\\
	$y^{\text{СДНФ}}_c = \bar{x_0} \bar{x_2} \vee x_0 \bar{x_1} \bar{x_2} \vee x_0 \bar{x_1} \bar{x_3} \vee \bar{x_0} x_2 x_3 \vee \bar{x_0} x_1 x_2$
	\\
	$y^{\text{СКНФ}}_c = (x_0 \vee x_1 \vee \bar{x_2} \vee x_3)$
	\\      
	}

	\centering{
	\begin{tikzpicture}[karnaugh,
		American style, 
		grp/.style n args={3}{#1,fill=#1!30,
			minimum width=#2\kmunitlength,
			minimum height=#3\kmunitlength,
			rounded corners=0.2\kmunitlength,
			fill opacity=0.6,
			rectangle,draw}]
		\karnaughmap{4}{$d^{\text{СДНФ}}$}{{$x_2$}{$x_0$}{$x_3$}{$x_1$}}%
		{0000 0111 0101 1001}{
			\node[grp={blue}{0.9}{1.9}](n000) at (2.5,2.0) {};
			\node[grp={green}{0.9}{1.9}](n001) at (2.5,3.0) {};
			\node[grp={red}{1.9}{0.9}](n002) at (3.0,2.5) {};
			\node[grp={yellow}{0.9}{1.9}](n003) at (1.5,1.0) {};
		}
	\end{tikzpicture}
	\begin{tikzpicture}[karnaugh, American style]
		\karnaughmap{4}{$d^{\text{СКНФ}}$}{{$x_2$}{$x_0$}{$x_3$}{$x_1$}}%
		{1011 1101 0110 1111}{}
	\end{tikzpicture}
	\\
	$y^{\text{СДНФ}}_d = x_0 \bar{x_1} \bar{x_2} \vee \bar{x_0} \bar{x_1} x_2 \vee \bar{x_0} x_2 \bar{x_3} \vee \bar{x_0} \bar{x_1} \bar{x_3} \vee \bar{x_0} x_1 \bar{x_2} x_3$ 
	\\
	$y^{\text{СКНФ}}_d = (x_0 \vee x_1 \vee x_2 \vee \bar{x_3}) \cdot (x_0 \vee \bar{x_1} \vee x_2 \vee x_3) \cdot (x_0 \vee \bar{x_1} \vee \bar{x_2} \vee \bar{x_3}) \cdot (\bar{x_0} \vee x_1  \vee \bar{x_2} \vee x_3)$
	\\
	}

	\centering{
	\begin{tikzpicture}[karnaugh,
		American style, 
		grp/.style n args={3}{#1,fill=#1!30,
			minimum width=#2\kmunitlength,
			minimum height=#3\kmunitlength,
			rounded corners=0.2\kmunitlength,
			fill opacity=0.6,
			rectangle,draw}]
		\karnaughmap{4}{$e^{\text{СДНФ}}$}{{$x_2$}{$x_0$}{$x_3$}{$x_1$}}%
		{0001 0011 0001 0001}{
			\node[grp={blue}{1.9}{1.9}](n000) at (2.0,2.0) {};
			\node[grp={green}{1.9}{0.9}](n001) at (3.0,2.5) {};
		}
	\end{tikzpicture}
	\begin{tikzpicture}[karnaugh,
		American style, 
		grp/.style n args={3}{#1,fill=#1!30,
			minimum width=#2\kmunitlength,
			minimum height=#3\kmunitlength,
			rounded corners=0.2\kmunitlength,
			fill opacity=0.6,
			rectangle,draw}]
		\karnaughmap{4}{$e^{\text{СКНФ}}$}{{$x_2$}{$x_0$}{$x_3$}{$x_1$}}%
		{1011 1111 0000 0111}{
			\node[grp={blue}{1.9}{1.9}](n000) at (1.0,1.0) {};
			\node[grp={green}{0.9}{0.9}](n001) at (1.5,3.5) {};
			\node[grp={green}{0.9}{0.9}](n001) at (1.5,0.5) {};
			\node[grp={red}{0.9}{0.9}](n002) at (0.5,0.5) {};
			\node[grp={red}{0.9}{0.9}](n002) at (3.5,0.5) {};
		}
	\end{tikzpicture}
	\\
	$y^{\text{СДНФ}}_e = \bar{x_1} \bar{x_3} \vee \bar{x_0} x_2 \bar{x_3}$ 
	\\
	$y^{\text{СКНФ}}_e = (x_0 \vee \bar{x_3}) \cdot (x_0 \vee \bar{x_1} \vee x_2) \cdot (x_1 \vee x_2 \vee \bar{x_3})$
	\\
	}
	\centering{
	\begin{tikzpicture}[karnaugh,
		American style, 
		grp/.style n args={3}{#1,fill=#1!30,
			minimum width=#2\kmunitlength,
			minimum height=#3\kmunitlength,
			rounded corners=0.2\kmunitlength,
			fill opacity=0.6,
			rectangle,draw}]
		\karnaughmap{4}{$f^{\text{СДНФ}}$}{{$x_2$}{$x_0$}{$x_3$}{$x_1$}}%
		{0001 0010 0101 1011}{
			\node[grp={blue}{0.9}{1.9}](n000) at (1.5,2.0) {};
			\node[grp={green}{0.9}{1.9}](n001) at (1.5,1.0) {};
			\node[grp={red}{1.9}{0.9}](n002) at (3.0,1.5) {};
			\node[grp={yellow}{0.9}{1.9}](n003) at (3.5,2.0) {};
			\node[grp={cyan}{0.9}{1.9}](n004) at (3.5,1.0) {};
		}
	\end{tikzpicture}
	\begin{tikzpicture}[karnaugh,
		American style, 
		grp/.style n args={3}{#1,fill=#1!30,
			minimum width=#2\kmunitlength,
			minimum height=#3\kmunitlength,
			rounded corners=0.2\kmunitlength,
			fill opacity=0.6,
			rectangle,draw}]
		\karnaughmap{4}{$f^{\text{СКНФ}}$}{{$x_2$}{$x_0$}{$x_3$}{$x_1$}}%
		{1101 1111 0100 1111}{
			\node[grp={blue}{0.9}{1.9}](n000) at (0.5,2.0) {};
			\node[grp={green}{0.9}{1.9}](n001) at (0.5,1.0) {};
			\node[grp={red}{1.9}{0.9}](n002) at (1.0,1.5) {};
		}
	\end{tikzpicture}
	\\
	$y^{\text{СДНФ}}_f = x_0 \bar{x_1} \bar{x_3} \vee x_0 \bar{x_1} \bar{x_2} \vee \bar{x_0} \bar{x_2} \bar{x_3} \vee \bar{x_0} x_1 x_3 \vee \bar{x_0} x_1 \bar{x_2}$ 
	\\
	$y^{\text{СКНФ}}_f = (x_0 \vee x_1 \vee \bar{x_2}) \cdot (x_0 \vee \bar{x_2} \vee \bar{x_3}) \cdot (x_0 \vee x_1 \vee \bar{x_3})$
	\\
	}


	\centering{
	\begin{tikzpicture}[karnaugh,
		American style, 
		grp/.style n args={3}{#1,fill=#1!30,
			minimum width=#2\kmunitlength,
			minimum height=#3\kmunitlength,
			rounded corners=0.2\kmunitlength,
			fill opacity=0.6,
			rectangle,draw}]
		\karnaughmap{4}{$g^{\text{СДНФ}}$}{{$x_2$}{$x_0$}{$x_3$}{$x_1$}}%
		{0001 0111 0001 1010}{
			\node[grp={blue}{0.9}{1.9}](n000) at (1.5,2.0) {};
			\node[grp={green}{0.9}{1.9}](n001) at (2.5,3.0) {};
			\node[grp={red}{0.9}{1.9}](n002) at (3.5,2.0) {};
			\node[grp={yellow}{0.9}{1.9}](n003) at (3.5,1.0) {};
		}
	\end{tikzpicture}
	\begin{tikzpicture}[karnaugh,
		American style, 
		grp/.style n args={3}{#1,fill=#1!30,
			minimum width=#2\kmunitlength,
			minimum height=#3\kmunitlength,
			rounded corners=0.2\kmunitlength,
			fill opacity=0.6,
			rectangle,draw}]
		\karnaughmap{4}{$g^{\text{СКНФ}}$}{{$x_2$}{$x_0$}{$x_3$}{$x_1$}}%
		{0111 1101 0110 1111}{
			\node[grp={blue}{0.9}{0.9}](n000) at (0.5,3.5) {};
			\node[grp={blue}{0.9}{0.9}](n000) at (0.5,0.5) {};
			%\node[grp={green}{0.9}{1.9}](n001) at (2.5,3.0) {};
			%\node[grp={red}{0.9}{1.9}](n002) at (3.5,2.0) {};
			%\node[grp={yellow}{0.9}{1.9}](n003) at (3.5,1.0) {};
		}
	\end{tikzpicture}
	\\
	$y^{\text{СДНФ}}_g = x_0 \bar{x_1} \bar{x_3} \vee \bar{x_0} \bar{x_1} x_2 \vee \bar{x_0} x_1 \bar{x_3} \vee \bar{x_0} x_1 \bar{x_2}$ 
	\\
	$y^{\text{СКНФ}}_g = (x_0 \vee x_1 \vee x_2) \cdot (x_0 \vee \bar{x_1} \vee \bar{x_2} \vee \bar{x_3}) \cdot (\bar{x_0} \vee x_1 \vee \bar{x_2} \vee x_3)$
	\\
	}

	\subsection{Метод Квайна}
	\subsubsection{СДНФ}
	\begin{center}
		\begin{tabular}{ |c|c|c|c| }     
		    \hline
		$\bar{x_0} \bar{x_1} \bar{x_2} \bar{x_3} $  0&0 + 1 = $\bar{x_0} \bar{x_1} \bar{x_3} $  0' &0' + 9' = $\bar{x_1} \bar{x_3} $  0'' &$\bar{x_0} x_1 x_3 $\\
		$\bar{x_0} \bar{x_1} x_2 \bar{x_3} $  1&0 + 6 = $\bar{x_1} \bar{x_2} \bar{x_3} $  1' &1' + 4' = $\bar{x_1} \bar{x_3} $  1'' &$\bar{x_0} x_2 $\\
		$\bar{x_0} \bar{x_1} x_2 x_3 $  2&1 + 2 = $\bar{x_0} \bar{x_1} x_2 $  2' &2' + 7' = $\bar{x_0} x_2 $  2'' &$\bar{x_1} \bar{x_3} $\\
		$\bar{x_0} x_1 \bar{x_2} x_3 $  3&1 + 4 = $\bar{x_0} x_2 \bar{x_3} $  3' &3' + 5' = $\bar{x_0} x_2 $  3'' &$x_0 \bar{x_1} \bar{x_2} $\\
		$\bar{x_0} x_1 x_2 \bar{x_3} $  4&1 + 8 = $\bar{x_1} x_2 \bar{x_3} $  4' & &   \\
		$\bar{x_0} x_1 x_2 x_3 $  5&2 + 5 = $\bar{x_0} x_2 x_3 $  5' & &   \\
		$x_0 \bar{x_1} \bar{x_2} \bar{x_3} $  6&3 + 5 = $\bar{x_0} x_1 x_3 $  6' & &   \\
		$x_0 \bar{x_1} \bar{x_2} x_3 $  7&4 + 5 = $\bar{x_0} x_1 x_2 $  7' & &   \\
		$x_0 \bar{x_1} x_2 \bar{x_3} $  8&6 + 7 = $x_0 \bar{x_1} \bar{x_2} $  8' & &   \\
		&6 + 8 = $x_0 \bar{x_1} \bar{x_3} $  9' & &   \\
			\hline
		\end{tabular}
	\end{center}
	$y_a =$ $\bar{x_0} x_1 x_3 $ + $\bar{x_0} x_2 $ + $\bar{x_1} \bar{x_3} $ + $x_0 \bar{x_1} \bar{x_2} $


	\begin{center}
		\begin{tabular}{ |c|c|c|c| }         
			\hline
		$\bar{x_0} \bar{x_1} \bar{x_2} \bar{x_3} $  0&0 + 1 = $\bar{x_0} \bar{x_1} \bar{x_2} $  0' &0' + 6' = $\bar{x_0} \bar{x_1} $  0'' &$\bar{x_0} \bar{x_1} $\\
		$\bar{x_0} \bar{x_1} \bar{x_2} x_3 $  1&0 + 2 = $\bar{x_0} \bar{x_1} \bar{x_3} $  1' &0' + 9' = $\bar{x_1} \bar{x_2} $  1'' &$\bar{x_0} \bar{x_2} \bar{x_3} $\\
		$\bar{x_0} \bar{x_1} x_2 \bar{x_3} $  2&0 + 4 = $\bar{x_0} \bar{x_2} \bar{x_3} $  2' &1' + 4' = $\bar{x_0} \bar{x_1} $  2'' &$\bar{x_0} x_2 x_3 $\\
		$\bar{x_0} \bar{x_1} x_2 x_3 $  3&0 + 6 = $\bar{x_1} \bar{x_2} \bar{x_3} $  3' &1' + 10' = $\bar{x_1} \bar{x_3} $  3'' &$\bar{x_1} \bar{x_2} $\\
		$\bar{x_0} x_1 \bar{x_2} \bar{x_3} $  4&1 + 3 = $\bar{x_0} \bar{x_1} x_3 $  4' &3' + 5' = $\bar{x_1} \bar{x_2} $  4'' &$\bar{x_1} \bar{x_3} $\\
		$\bar{x_0} x_1 x_2 x_3 $  5&1 + 7 = $\bar{x_1} \bar{x_2} x_3 $  5' &3' + 7' = $\bar{x_1} \bar{x_3} $  5'' &   \\
		$x_0 \bar{x_1} \bar{x_2} \bar{x_3} $  6&2 + 3 = $\bar{x_0} \bar{x_1} x_2 $  6' & &   \\
		$x_0 \bar{x_1} \bar{x_2} x_3 $  7&2 + 8 = $\bar{x_1} x_2 \bar{x_3} $  7' & &   \\
		$x_0 \bar{x_1} x_2 \bar{x_3} $  8&3 + 5 = $\bar{x_0} x_2 x_3 $  8' & &   \\
		&6 + 7 = $x_0 \bar{x_1} \bar{x_2} $  9' & &   \\
		&6 + 8 = $x_0 \bar{x_1} \bar{x_3} $ 10' & &   \\
			\hline
		\end{tabular}
	\end{center}
	$y_b =$ $\bar{x_0} \bar{x_1} $ + $\bar{x_0} \bar{x_2} \bar{x_3} $ + $\bar{x_0} x_2 x_3 $ + $\bar{x_1} \bar{x_2} $ + $\bar{x_1} \bar{x_3} $


	\begin{center}
		\begin{tabular}{ |c|c|c|c| }   
		    \hline
		$\bar{x_0} \bar{x_1} \bar{x_2} \bar{x_3} $  0&0 + 1 = $\bar{x_0} \bar{x_1} \bar{x_2} $  0' &0' + 7' = $\bar{x_0} \bar{x_2} $  0'' &$\bar{x_0} \bar{x_2} $\\
		$\bar{x_0} \bar{x_1} \bar{x_2} x_3 $  1&0 + 3 = $\bar{x_0} \bar{x_2} \bar{x_3} $  1' &0' + 11' = $\bar{x_1} \bar{x_2} $  1'' &$\bar{x_0} x_1 $\\
		$\bar{x_0} \bar{x_1} x_2 x_3 $  2&0 + 7 = $\bar{x_1} \bar{x_2} \bar{x_3} $  2' &1' + 4' = $\bar{x_0} \bar{x_2} $  2'' &$\bar{x_0} x_3 $\\
		$\bar{x_0} x_1 \bar{x_2} \bar{x_3} $  3&1 + 2 = $\bar{x_0} \bar{x_1} x_3 $  3' &2' + 5' = $\bar{x_1} \bar{x_2} $  3'' &$\bar{x_1} \bar{x_2} $\\
		$\bar{x_0} x_1 \bar{x_2} x_3 $  4&1 + 4 = $\bar{x_0} \bar{x_2} x_3 $  4' &3' + 9' = $\bar{x_0} x_3 $  4'' &$x_0 \bar{x_1} \bar{x_3} $\\
		$\bar{x_0} x_1 x_2 \bar{x_3} $  5&1 + 8 = $\bar{x_1} \bar{x_2} x_3 $  5' &4' + 6' = $\bar{x_0} x_3 $  5'' &   \\
		$\bar{x_0} x_1 x_2 x_3 $  6&2 + 6 = $\bar{x_0} x_2 x_3 $  6' &7' + 10' = $\bar{x_0} x_1 $  6'' &   \\
		$x_0 \bar{x_1} \bar{x_2} \bar{x_3} $  7&3 + 4 = $\bar{x_0} x_1 \bar{x_2} $  7' &8' + 9' = $\bar{x_0} x_1 $  7'' &   \\
		$x_0 \bar{x_1} \bar{x_2} x_3 $  8&3 + 5 = $\bar{x_0} x_1 \bar{x_3} $  8' & &   \\
		$x_0 \bar{x_1} x_2 \bar{x_3} $  9&4 + 6 = $\bar{x_0} x_1 x_3 $  9' & &   \\
		&5 + 6 = $\bar{x_0} x_1 x_2 $ 10' & &   \\
		&7 + 8 = $x_0 \bar{x_1} \bar{x_2} $ 11' & &   \\
		&7 + 9 = $x_0 \bar{x_1} \bar{x_3} $ 12' & &   \\
			\hline
		\end{tabular}
	\end{center}
	$y_c =$ $\bar{x_0} \bar{x_2} $ + $\bar{x_0} x_1 $ + $\bar{x_0} x_3 $ + $\bar{x_1} \bar{x_2} $ + $x_0 \bar{x_1} \bar{x_3} $


	\begin{center}
		\begin{tabular}{ |c|c|c| }       
			\hline
		$\bar{x_0} \bar{x_1} \bar{x_2} \bar{x_3} $  0&0 + 1 = $\bar{x_0} \bar{x_1} \bar{x_3} $  0' &$\bar{x_0} \bar{x_1} \bar{x_3} $\\
		$\bar{x_0} \bar{x_1} x_2 \bar{x_3} $  1&0 + 5 = $\bar{x_1} \bar{x_2} \bar{x_3} $  1' &$\bar{x_0} \bar{x_1} x_2 $\\
		$\bar{x_0} \bar{x_1} x_2 x_3 $  2&1 + 2 = $\bar{x_0} \bar{x_1} x_2 $  2' &$\bar{x_0} x_2 \bar{x_3} $\\
		$\bar{x_0} x_1 \bar{x_2} x_3 $  3&1 + 4 = $\bar{x_0} x_2 \bar{x_3} $  3' &$\bar{x_1} \bar{x_2} \bar{x_3} $\\
		$\bar{x_0} x_1 x_2 \bar{x_3} $  4&5 + 6 = $x_0 \bar{x_1} \bar{x_2} $  4' &$x_0 \bar{x_1} \bar{x_2} $\\
		$x_0 \bar{x_1} \bar{x_2} \bar{x_3} $  5& &   \\
		$x_0 \bar{x_1} \bar{x_2} x_3 $  6& &   \\
			\hline
		\end{tabular}
	\end{center}
	$y_d =$ $\bar{x_0} \bar{x_1} \bar{x_3} $ + $\bar{x_0} \bar{x_1} x_2 $ + $\bar{x_0} x_2 \bar{x_3} $ + $\bar{x_1} \bar{x_2} \bar{x_3} $ + $x_0 \bar{x_1} \bar{x_2} $


	\begin{center}
		\begin{tabular}{ |c|c|c|c| }
			\hline
		$\bar{x_0} \bar{x_1} \bar{x_2} \bar{x_3} $  0&0 + 1 = $\bar{x_0} \bar{x_1} \bar{x_3} $  0' &0' + 4' = $\bar{x_1} \bar{x_3} $  0'' &$\bar{x_0} x_2 \bar{x_3} $\\
		$\bar{x_0} \bar{x_1} x_2 \bar{x_3} $  1&0 + 3 = $\bar{x_1} \bar{x_2} \bar{x_3} $  1' &1' + 3' = $\bar{x_1} \bar{x_3} $  1'' &$\bar{x_1} \bar{x_3} $\\
		$\bar{x_0} x_1 x_2 \bar{x_3} $  2&1 + 2 = $\bar{x_0} x_2 \bar{x_3} $  2' & &   \\
		$x_0 \bar{x_1} \bar{x_2} \bar{x_3} $  3&1 + 4 = $\bar{x_1} x_2 \bar{x_3} $  3' & &   \\
		$x_0 \bar{x_1} x_2 \bar{x_3} $  4&3 + 4 = $x_0 \bar{x_1} \bar{x_3} $  4' & &   \\
			\hline
		\end{tabular}
	\end{center}
	$y_e =$ $\bar{x_0} x_2 \bar{x_3} $ + $\bar{x_1} \bar{x_3} $


	\begin{center}
		\begin{tabular}{ |c|c|c| }
			\hline
		$\bar{x_0} \bar{x_1} \bar{x_2} \bar{x_3} $  0&0 + 1 = $\bar{x_0} \bar{x_2} \bar{x_3} $  0' &$\bar{x_0} \bar{x_2} \bar{x_3} $\\
		$\bar{x_0} x_1 \bar{x_2} \bar{x_3} $  1&0 + 4 = $\bar{x_1} \bar{x_2} \bar{x_3} $  1' &$\bar{x_0} x_1 \bar{x_2} $\\
		$\bar{x_0} x_1 \bar{x_2} x_3 $  2&1 + 2 = $\bar{x_0} x_1 \bar{x_2} $  2' &$\bar{x_0} x_1 \bar{x_3} $\\
		$\bar{x_0} x_1 x_2 \bar{x_3} $  3&1 + 3 = $\bar{x_0} x_1 \bar{x_3} $  3' &$\bar{x_1} \bar{x_2} \bar{x_3} $\\
		$x_0 \bar{x_1} \bar{x_2} \bar{x_3} $  4&4 + 5 = $x_0 \bar{x_1} \bar{x_2} $  4' &$x_0 \bar{x_1} \bar{x_2} $\\
		$x_0 \bar{x_1} \bar{x_2} x_3 $  5&4 + 6 = $x_0 \bar{x_1} \bar{x_3} $  5' &$x_0 \bar{x_1} \bar{x_3} $\\
		$x_0 \bar{x_1} x_2 \bar{x_3} $  6& &   \\
			\hline
		\end{tabular}
	\end{center}
	$y_f =$ $\bar{x_0} \bar{x_2} \bar{x_3} $ + $\bar{x_0} x_1 \bar{x_2} $ + $\bar{x_0} x_1 \bar{x_3} $ + $\bar{x_1} \bar{x_2} \bar{x_3} $ + $x_0 \bar{x_1} \bar{x_2} $ + $x_0 \bar{x_1} \bar{x_3} $


	\begin{center}
		\begin{tabular}{ |c|c|c| }
			\hline
		$\bar{x_0} \bar{x_1} x_2 \bar{x_3} $  0&0 + 1 = $\bar{x_0} \bar{x_1} x_2 $  0' &$\bar{x_0} \bar{x_1} x_2 $\\
		$\bar{x_0} \bar{x_1} x_2 x_3 $  1&0 + 4 = $\bar{x_0} x_2 \bar{x_3} $  1' &$\bar{x_0} x_1 \bar{x_2} $\\
		$\bar{x_0} x_1 \bar{x_2} \bar{x_3} $  2&2 + 3 = $\bar{x_0} x_1 \bar{x_2} $  2' &$\bar{x_0} x_1 \bar{x_3} $\\
		$\bar{x_0} x_1 \bar{x_2} x_3 $  3&2 + 4 = $\bar{x_0} x_1 \bar{x_3} $  3' &$\bar{x_0} x_2 \bar{x_3} $\\
		$\bar{x_0} x_1 x_2 \bar{x_3} $  4&5 + 6 = $x_0 \bar{x_1} \bar{x_2} $  4' &$x_0 \bar{x_1} \bar{x_2} $\\
		$x_0 \bar{x_1} \bar{x_2} \bar{x_3} $  5& &   \\
		$x_0 \bar{x_1} \bar{x_2} x_3 $  6& &   \\
			\hline
		\end{tabular}
	\end{center}
	$y_g =$ $\bar{x_0} \bar{x_1} x_2 $ + $\bar{x_0} x_1 \bar{x_2} $ + $\bar{x_0} x_1 \bar{x_3} $ + $\bar{x_0} x_2 \bar{x_3} $ + $x_0 \bar{x_1} \bar{x_2} $

	\subsubsection{СКНФ}

	Корректно сократить возможно только СКНФ для e, f и g.
	\begin{center}
		\begin{tabular}[h]{ |c|c|c|c| }         
			\hline
$(x_0\vee x_1\vee x_2\vee\bar{x_3})$ 		       0&0+1=$(x_0 \vee x_1 \vee \bar{x_3})$ 0'	     &0'+5'=$(x_0 \vee \bar{x_3} )$ 0''&$(x_0 \vee \bar{x_1} \vee x_2 )$\\
$(x_0\vee x_1\vee\bar{x_2}\vee\bar{x_3})$ 	       1&0+3=$(x_0 \vee x_2 \vee \bar{x_3})$ 1'	     &1'+3'=$(x_0 \vee \bar{x_3} )$ 1''&$(x_0 \vee \bar{x_3} )$         \\
$(x_0\vee\bar{x_1} \vee x_2 \vee x_3)$ 			   2&0+5=$(x_1 \vee x_2 \vee \bar{x_3})$ 2' 	 & 								   &$(x_1 \vee x_2 \vee \bar{x_3} )$\\
$(x_0\vee\bar{x_1} \vee x_2 \vee \bar{x_3})$ 	   3&1+4=$(x_0 \vee \bar{x_2} \vee \bar{x_3})$ 3'& 								   &   							    \\
$(x_0\vee\bar{x_1} \vee \bar{x_2} \vee \bar{x_3})$ 4&2+3=$(x_0 \vee \bar{x_1} \vee x_2)$ 4'      & 								   &   							    \\
$(\bar{x_0} \vee x_1 \vee x_2 \vee \bar{x_3})$     5&3+4=$(x_0 \vee \bar{x_1} \vee \bar{x_3})$ 5'& 								   &  							    \\
			\hline
		\end{tabular}
	\end{center}
	$y_e =$ $(x_0 \vee \bar{x_1} \vee x_2 )$ $\cdot$ $(x_0 \vee \bar{x_3} )$ $\cdot$ $(x_1 \vee x_2 \vee \bar{x_3} )$


	\begin{center}
		\begin{tabular}{ |c|c|c| }
			\hline
$(x_0\vee x_1\vee x_2\vee\bar{x_3})$ 0			&0+2=$(x_0\vee x_1\vee\bar{x_3})$ 0' 	  &$(x_0\vee\bar{x_2}\vee\bar{x_3})$\\
$(x_0\vee x_1\vee\bar{x_2}\vee x_3)$ 1			&1+2=$(x_0\vee x_1\vee\bar{x_2})$ 1' 	  &$(x_0\vee x_1\vee\bar{x_2})$     \\
$(x_0\vee x_1\vee\bar{x_2}\vee\bar{x_3})$ 2		&2+3=$(x_0\vee\bar{x_2}\vee\bar{x_3})$ 2' &$(x_0\vee x_1\vee\bar{x_3})$     \\
$(x_0\vee\bar{x_1}\vee\bar{x_2}\vee\bar{x_3})$ 3& 										  &  								\\
			\hline
		\end{tabular}
	\end{center}
	$y_f =$ $(x_0 \vee \bar{x_2} \vee \bar{x_3} )$ $\cdot$ $(x_0 \vee x_1 \vee \bar{x_2} )$ $\cdot$ $(x_0 \vee x_1 \vee \bar{x_3} )$


	\begin{center}
		\begin{tabular}{ |c|c|c| }
			\hline
$(x_0\vee x_1\vee x_2\vee x_3)$ 0				 &0+1=$(x_0\vee x_1\vee x_2)$ 0' &$(x_0 \vee x_1 \vee x_2 )$ 					\\
$(x_0\vee x_1\vee x_2\vee\bar{x_3})$ 1			 & 							     &$(x_0\vee\bar{x_1}\vee\bar{x_2}\vee\bar{x_3})$\\
$(x_0\vee\bar{x_1}\vee\bar{x_2}\vee\bar{x_3})$ 2 & 								 &$(\bar{x_0}\vee x_1\vee\bar{x_2}\vee x_3)$    \\
$(\bar{x_0}\vee x_1\vee\bar{x_2}\vee x_3)$ 3	 & 								 &   				   	    					\\
			\hline
		\end{tabular}
	\end{center}
	$y_g =$ $(x_0 \vee x_1 \vee x_2 ) \cdot $$(x_0\vee\bar{x_1}\vee\bar{x_2}\vee\bar{x_3})$ $\cdot(\bar{x_0}\vee x_1\vee\bar{x_2}\vee x_3)$

	\section{Перевод полученных выражений к базису 2И-НЕ/2ИЛИ-НЕ}
	При переводе в базис к изначальному алгебраическому уравнению применяется двойное отрицание, после чего используются законы де Морагана:
	
	\begin{center}
		$\overline{a \cdot b} = \bar{a} \vee \bar{b}$ \\
		$\overline{a \vee b} = \bar{a} \cdot \bar{b}$ \\
	\end{center}
	
	Чтобы не загромождать запись двойными отрицаниями, они будут опускаться после того, как будет показано их применение, то есть:
	
	\begin{center}
		$\overline{\overline{A \vee B \vee C \vee D}} =$
		$\overline{\bar{A} \cdot \bar{B} \cdot \bar{C} \cdot \bar{D}}$ \\
		$A \vee B \vee C \vee D =$
		$\overline{\overline{
				   \overline{\overline{A \vee B}} \vee
				   \overline{\overline{C \vee D}}
				}}$
	\end{center}

	\newpage

	\subsection{2И-НЕ}	
	$y^{\text{ДНФ}}_a = \overline{\overline{\bar{x_0} x_2 \vee \bar{x_1} \bar{x_3} \vee x_0 \bar{x_1} \bar{x_2} \vee \bar{x_0} x_1 x_2}} =$ \\
	$= \overline{\overline{\bar{x_0} x_2} \cdot \overline{\bar{x_0} \bar{x_3}} \cdot \overline{x_0 \bar{x_1} \bar{x_2}} \cdot \overline{\bar{x_0} x_1 x_2}}$ \\
	$y^{\text{КНФ}}_a = \overline{\overline{\overline{\overline{(x_0 \vee x_1 \vee x_2 \vee \bar{x_3})}} \cdot (x_0 \vee \bar{x_1} \vee x_2 \vee x_3)}} = $ \\
	$= \overline{\overline{(x_0 \vee x_1 \vee x_2 \vee \bar{x_3})}} \cdot \overline{\overline{(x_0 \vee \bar{x_1} \vee x_2 \vee x_3)}} = $ \\
	$= \overline{\bar{x_0} \bar{x_1} \bar{x_2} x_3} \cdot \overline{\bar{x_0} x_1 \bar{x_2} \bar{x_3}}$ \\
	
	$y^{\text{ДНФ}}_b = \overline{\overline{\bar{x_1} x_2 \bar{x_3} \vee \bar{x_0} x_2 x_3 \vee \bar{x_0} \bar{x_1} x_3 \vee \bar{x_0} \bar{x_2} \bar{x_3} \vee x_0 \bar{x_1} \bar{x_3}}} = $ \\
	$= \overline{\overline{\bar{x_1} x_2 \bar{x_3}} \cdot\overline{\bar{x_0} x_2 x_3} \cdot\overline{\bar{x_0} \bar{x_1} x_3} \cdot\overline{\bar{x_0} \bar{x_2} \bar{x_3}} \cdot \overline{x_0 \bar{x_1} \bar{x_3}}}$ \\
	$y^{\text{КНФ}}_b = \overline{\overline{\overline{\overline{(x_0 \vee \bar{x_1} \vee x_2 \bar{x_3})}} \cdot(x_0 \vee \bar{x_1} \vee \bar{x_2} \vee x_3) =}}$ \\
	$= \overline{\overline{(x_0 \vee \bar{x_1} \vee x_2 \bar{x_3})}} \cdot\overline{\overline{(x_0 \vee \bar{x_1} \vee \bar{x_2} \vee x_3)}} =$ \\
	$= \overline{\bar{x_0} x_1 \bar{x_2} x_3} \cdot \overline{\bar{x_0} x_1 x_2 \bar{x_3}}$ \\
	
	$y^{\text{ДНФ}}_c = \overline{\overline{\bar{x_0} \bar{x_2} \vee x_0 \bar{x_1} \bar{x_2} \vee x_0 \bar{x_1} \bar{x_3} \vee\bar{x_0} x_2 x_3 \vee\bar{x_0} x_1 x_2}} =$ \\
	$= \overline{\overline{\bar{x_0} \bar{x_2}} \cdot\overline{x_0 \bar{x_1} \bar{x_2}} \cdot\overline{x_0 \bar{x_1} \bar{x_3}} \cdot\overline{\bar{x_0} x_2 x_3} \cdot \overline{\bar{x_0} x_1 x_2}}$ \\
	$y^{\text{КНФ}}_c = \overline{\overline{(x_0 \vee x_1 \vee \bar{x_2} \vee x_3)}} =$ $ \overline{\bar{x_0} \bar{x_1} x_2 \bar{x_3}}$ \\
	
	$y^{\text{ДНФ}}_d = \overline{\overline{x_0 \bar{x_1} \bar{x_2} \vee\bar{x_0} \bar{x_1} x_2 \vee\bar{x_0} x_2 \bar{x_3} \vee\bar{x_0} \bar{x_1} \bar{x_3} \vee\bar{x_0} x_1 \bar{x_2} x_3}} =$ \\
	$\overline{\overline{x_0 \bar{x_1} \bar{x_2}} \cdot \overline{\bar{x_0} \bar{x_1} x_2} \cdot \overline{\bar{x_0} x_2 \bar{x_3}} \cdot \overline{\bar{x_0} \bar{x_1} \bar{x_3}} \cdot \overline{\bar{x_0} x_1 \bar{x_2} x_3}}$ \\
	$y^{\text{КНФ}}_d = \overline{\overline{\overline{\overline{(x_0 \vee x_1 \vee x_2 \vee \bar{x_3}) \cdot(x_0 \vee \bar{x_1} \vee x_2 \vee x_3) \cdot(x_0 \vee \bar{x_1} \vee \bar{x_2} \vee \bar{x_3})}} \cdot (\bar{x_0} \vee x_1 \vee \bar{x_2} \vee x_3)}}$ \\
	$= \overline{\overline{(x_0 \vee x_1 \vee x_2 \vee \bar{x_3})}} \cdot \overline{\overline{(x_0 \vee \bar{x_1} \vee x_2 \vee x_3)}} \cdot\overline{\overline{(x_0 \vee \bar{x_1} \vee \bar{x_2} \vee \bar{x_3})}} \cdot \overline{\overline{(\bar{x_0} \vee x_1 \vee \bar{x_2} \vee x_3)}}$ \\
	$= \overline{\bar{x_0} \bar{x_1} \bar{x_2} x_3} \cdot \overline{\bar{x_0} x_1 \bar{x_2} \bar{x_3}} \cdot \overline{\bar{x_0} x_1 x_2 x_3} \cdot \overline{x_0 \bar{x_1} x_2 \bar{x_3}}$ \\
	
	$y^{\text{ДНФ}}_e = \overline{\overline{\bar{x_1} \bar{x_3} \vee \bar{x_0} x_2 \bar{x_3}}} = \overline{\overline{\bar{x_1} \bar{x_3}} \cdot \overline{\bar{x_0} x_2 \bar{x_3}}}$ \\
	$y^{\text{КНФ}}_e = \overline{\overline{\overline{\overline{(x_0 \vee \bar{x_3}) \vee (x_0 \vee \bar{x_1} \vee x_2)}} \cdot (x_1 \vee x_2 \vee \bar{x_3})}} =$ \\
	$= \overline{\overline{(x_0 \vee \bar{x_3})}} \cdot \overline{\overline{(x_0 \vee \bar{x_1} \vee x_2)}} \cdot \overline{\overline{(x_1 \vee x_2 \vee \bar{x_3})}} =$ \\
	$= \overline{\bar{x_0} x_3} \cdot \overline{\bar{x_0} x_1 \bar{x_2}} \cdot \overline{\bar{x_1} \bar{x_2} x_3}$ \\
	
	$y^{\text{ДНФ}}_f = \overline{\overline{x_0 \bar{x_1} \bar{x_3} \vee x_0 \bar{x_1} \bar{x_2} \vee \bar{x_0} \bar{x_2} \bar{x_3} \vee \bar{x_0} x_1 x_2 \vee \bar{x_0} x_1 \bar{x_2}}} =$ \\
	$= \overline{\overline{x_0 \bar{x_1} \bar{x_3}} \cdot \overline{x_0 \bar{x_1} \bar{x_2}} \cdot \overline{\bar{x_0} \bar{x_2} \bar{x_3}} \cdot \overline{\bar{x_0} x_1 x_3} \cdot \overline{\bar{x_0} x_1 \bar{x_2}}}$ \\
	$y^{\text{КНФ}}_f = \overline{\overline{\overline{\overline{(x_0 \vee x_1 \vee \bar{x_2}) \cdot(x_0 \vee \bar{x_2} \vee \bar{x_3})}} \cdot (x_0 \vee x_1 \vee \bar{x_3})}} =$ \\
	$= \overline{\overline{(x_0 \vee x_1 \vee \bar{x_2})}} \cdot \overline{\overline{(x_0 \vee \bar{x_2} \vee \bar{x_3})}} \cdot \overline{\overline{(x_0 \vee x_1 \vee \bar{x_3})}}$
	$= \overline{\bar{x_0} \bar{x_1} x_2} \cdot \overline{\bar{x_0} x_2 x_3} \cdot \overline{\bar{x_0} \bar{x_1} x_3}$ \\
	
	$y^{\text{ДНФ}}_g = \overline{\overline{x_0 \bar{x_1} \bar{x_3} \vee \bar{x_0} \bar{x_1} x_2 \vee \bar{x_0} x_1 \bar{x_3} \vee \bar{x_0} x_1 \bar{x_2}}}$
	$= \overline{\overline{x_0 \bar{x_1} \bar{x_3}} \cdot \overline{\bar{x_0} \bar{x_1} x_2} \cdot \overline{\bar{x_0} x_1 \bar{x_3}} \cdot \overline{\bar{x_0} x_1 \bar{x_2}}}$ \\
	$y^{\text{КНФ}}_g = \overline{\overline{\overline{\overline{(x_0 \vee x_1 \vee x_2) \cdot (x_0 \vee \bar{x_1} \vee \bar{x_2} \vee \bar{x_3})}} \cdot (\bar{x_0} \vee x_1 \vee \bar{x_2} \vee x_3)}} =$ \\
	$= \overline{\overline{(x_0 \vee x_1 \vee x_2)}} \cdot \overline{\overline{(x_0 \vee \bar{x_1} \vee \bar{x_2} \vee \bar{x_3})}} \cdot \overline{\overline{(\bar{x_0} \vee x_1 \vee \bar{x_2} \vee x_3)}} =$ \\
	$= \overline{\bar{x_0} \bar{x_1} \bar{x_2}} \cdot \overline{\bar{x_0} x_1 x_2 x_3} \cdot \overline{x_0 \bar{x_1} x_2 \bar{x_3}}$ \\
	
	\subsection{2ИЛИ-НЕ}
	
	$y^{\text{ДНФ}}_a = \overline{\overline{\overline{\overline{ \bar{x_0} x_2 \vee \bar{x_1} \bar{x_3} }} \vee \overline{\overline{ x_0 \bar{x_1} \bar{x_2} \vee \bar{x_0} x_1 x_3}} }}$
	$= \overline{\overline{\bar{x_0} x_2}} \vee \overline{\overline{\bar{x_1} \bar{x_3}}} \vee \overline{\overline{x_0 \bar{x_1} \bar{x_2}}} \vee \overline{\overline{\bar{x_0} x_1 x_3}} =$ \\
	$= \overline{ x_0 \vee \bar{x_2}} \vee \overline{x_1 \vee x_3} \vee \overline{\bar{x_0} \vee x_1 \vee x_2} \vee \overline{x_0 \vee \bar{x_1} \vee \bar{x_3}}$ \\
	$y^{\text{КНФ}}_a = \overline{\overline{(x_0 \vee x_1 \vee x_2 \vee \bar{x_3}) \cdot (x_0 \vee \bar{x_1} \vee x_2 \vee x_3)}} =$\\
	$= \overline{\overline{x_0 \vee x_1 \vee x_2 \vee \bar{x_3}} \vee \overline{x_0 \vee \bar{x_1} \vee x_2 \vee x_3}}$ \\
	
	$y^{\text{ДНФ}}_b = \overline{\overline{
			\overline{\overline{
					\bar{x_1} x_2 \bar{x_3} \vee \bar{x_0} x_2 x_3
			}} \vee
			\overline{\overline{
					\overline{\overline{
							\bar{x_0} \bar{x_1} x_3 \vee \bar{x_0} \bar{x_2} \bar{x_3}
					}} \vee x_0 \bar{x_1} \bar{x_3}
			}}
	}} =$ \\
	$= \overline{\overline{\bar{x_1} x_2 x_3}} \vee \overline{\overline{\bar{x_0} x_2 x_3}} \vee \overline{\overline{\bar{x_0} \bar{x_1} x_3}} \vee \overline{\overline{\bar{x_0} \bar{x_2} \bar{x_3}}} \vee \overline{\overline{x_0 \bar{x_1} \bar{x_3}}} =$ \\
	$= \overline{x_1 \vee \bar{x_2} \vee x_3} \vee \overline{x_0 \vee \bar{x_2} \vee \bar{x_3}} \vee \overline{x_0 \vee x_1 \vee \bar{x_3}} \vee \overline{x_0 \vee x_2 \vee x_3} \vee \overline{\bar{x_0} \vee x_1 \vee x_3}$ \\
	$y^{\text{КНФ}}_b = \overline{\overline{(x_0 \vee \bar{x_1} \vee x_2 \vee \bar{x_3}) \cdot (x_0 \vee \bar{x_1} \vee \bar{x_2} \vee x_3)}}$ \\
	$= \overline{\overline{x_0 \vee \bar{x_1} \vee x_2 \vee \bar{x_3}} \vee \overline{x_0 \vee \bar{x_1} \vee \bar{x_2} \vee x_3}}$ \\
	
	$y^{\text{ДНФ}}_c =  \overline{\overline{
			\overline{\overline{
					\bar{x_0} \bar{x_2} \vee x_0 \bar{x_1} \bar{x_2}
			}} \vee
			\overline{\overline{
					\overline{\overline{
							x_0 \bar{x_1} \bar{x_3} \vee \bar{x_0} x_2 x_3
					}} \vee \bar{x_0} x_1 x_2
			}}
	}} =$ \\
	$= \overline{\overline{\bar{x_0} \bar{x_2}}} \vee \overline{\overline{x_0 \bar{x_1} \bar{x_2}}} \vee  \overline{\overline{x_0 \bar{x_1} \bar{x_3}}} \vee  \overline{\overline{\bar{x_0} x_2 x_3}} \vee \overline{\overline{\bar{x_0} x_1 x_2}} =$ \\
	$= \overline{x_0 \vee x_2} \vee \overline{\bar{x_0} \vee x_1 \vee x_2} \vee \overline{\bar{x_0} \vee x_1 \vee x_3} \vee \overline{x_0 \vee \bar{x_2} \vee \bar{x_3}} \vee \overline{x_0 \vee \bar{x_1} \vee \bar{x_2}}$ \\
	$y^{\text{КНФ}}_c = \overline{\overline{x_0 \vee x_1 \vee \bar{x_2} \vee x_3}}$ \\
	
	$y^{\text{ДНФ}}_d = \overline{\overline{
			\overline{\overline{
					x_0 \bar{x_1} \bar{x_2} \vee \bar{x_0} \bar{x_1} x_2 
			}} \vee 
			\overline{\overline{
					\overline{\overline{
							\bar{x_0} x_2 \bar{x_3} \vee \bar{x_0} \bar{x_1} \bar{x_3}
					}} \vee \bar{x_0} x_1 \bar{x_2} x_3
			}}
	}} =$ \\
	$= \overline{\overline{x_0 \bar{x_1} \bar{x_2}}} \vee \overline{\overline{\bar{x_0} \bar{x_1} x_2}} \vee \overline{\overline{\bar{x_0} x_2 \bar{x_3}}} \vee \overline{\overline{\bar{x_0} \bar{x_1} \bar{x_3}}} \vee \overline{\overline{\bar{x_0} x_1 \bar{x_2} x_3}}$ \\
	$= \overline{\bar{x_0} \vee x_1 \vee x_2} \vee \overline{x_0 \vee x_1 \vee \bar{x_2}} \vee \overline{x_0 \vee \bar{x_2} \vee x_3} \vee \overline{x_0 \vee x_1 \vee x_3} \vee \overline{x_0 \vee \bar{x_1} \vee x_2 \vee \bar{x_3}}$ \\
	$y^{\text{КНФ}}_d = \overline{\overline{(x_0 \vee x_1 \vee x_2 \vee \bar{x_3}) \cdot (x_0 \vee \bar{x_1} \vee x_2 \vee x_3) \cdot (x_0 \vee \bar{x_1} \vee \bar{x_2} \vee \bar{x_3}) \cdot (\bar{x_0} \vee x_1  \vee \bar{x_2} \vee x_3)}} =$ \\
	$=\overline{\overline{x_0 \vee x_1 \vee x_2 \vee \bar{x_3}} \vee\overline{x_0 \vee \bar{x_1} \vee x_2 \vee x_3} \vee \overline{x_0 \vee \bar{x_1} \vee \bar{x_2} \vee \bar{x_3}} \vee \overline{\bar{x_0} \vee x_1  \vee \bar{x_2} \vee x_3}}$ \\
	
	$y^{\text{ДНФ}}_e = \overline{\overline{\bar{x_1} \bar{x_3} \vee \bar{x_0} x_2 \bar{x_3}}} $
	$= \overline{\overline{\bar{x_1} \bar{x_3}}} \vee \overline{\overline{\bar{x_0} x_2 \bar{x_3}}} $
	$= \overline{x_1 \vee x_3} \vee \overline{x_0 \vee \bar{x_2} \vee x_3}$ \\
	$y^{\text{КНФ}}_e = \overline{\overline{(x_0 \vee \bar{x_3}) \cdot (x_0 \vee \bar{x_1} \vee x_2) \cdot (x_1 \vee x_2 \vee \bar{x_3})}}$ \\
	$= \overline{\overline{x_0 \vee \bar{x_3}} \vee \overline{x_0 \vee \bar{x_1} \vee x_2} \vee \overline{x_1 \vee x_2 \vee \bar{x_3}}}$ \\
	
	$y^{\text{ДНФ}}_f = \overline{\overline{
			\overline{\overline{
					x_0 \bar{x_1} \bar{x_3} \vee x_0 \bar{x_1} \bar{x_2} 
			}} \vee
			\overline{\overline{
					\overline{\overline{
							\bar{x_0} \bar{x_2} \bar{x_3} \vee \bar{x_0} x_1 x_3 
					}} \vee \bar{x_0} x_1 \bar{x_2}
			}}
	}} =$ \\
	$= \overline{\overline{x_0 \bar{x_1} \bar{x_3}}} \vee \overline{\overline{x_0 \bar{x_1} \bar{x_2}}} \vee \overline{\overline{\bar{x_0} \bar{x_2} \bar{x_3}}} \vee \overline{\overline{\bar{x_0} x_1 x_3}} \vee \overline{\overline{\bar{x_0} x_1 \bar{x_2}}}$ \\
	$= \overline{\bar{x_0} \vee x_1 \vee x_3} \vee \overline{\bar{x_0} \vee x_1 \vee x_2} \vee \overline{x_0 \vee x_2 \vee x_3} \vee \overline{x_0 \vee \bar{x_1} \vee \bar{x_3}} \vee \overline{x_0 \vee \bar{x_1} \vee x_2}$ \\
	$y^{\text{КНФ}}_f = \overline{\overline{(x_0 \vee x_1 \vee \bar{x_2}) \cdot (x_0 \vee \bar{x_2} \vee \bar{x_3}) \cdot (x_0 \vee x_1 \vee \bar{x_3})}}$ \\
	$= \overline{ \overline{x_0 \vee x_1 \vee \bar{x_2}} \vee \overline{x_0 \vee \bar{x_2} \vee \bar{x_3}} \vee \overline{x_0 \vee x_1 \vee \bar{x_3}}}$ \\
	
	$y^{\text{ДНФ}}_g = \overline{\overline{\overline{\overline{\bar{x_0} \bar{x_1} \bar{x_3} \vee \bar{x_0} \bar{x_1} x_2}} \vee \overline{\overline{\bar{x_0} x_1 \bar{x_3} \vee \bar{x_0} x_1 \bar{x_2}}}}}$
	$= \overline{\overline{\bar{x_0} \bar{x_1} \bar{x_3}}} \vee
	\overline{\overline{\bar{x_0} \bar{x_1} x_2}} \vee
	\overline{\overline{\bar{x_0} x_1 \bar{x_3}}} \vee
	\overline{\overline{\bar{x_0} x_1 \bar{x_2}}} $ \\
	$= \overline{x_0 \vee x_1 \vee x_3} \vee \overline{x_0 \vee x_1 \vee \bar{x_2}} \vee \overline{x_0 \vee \bar{x_1} \vee x_3} \vee \overline{x_0 \vee \bar{x_1} \vee x_2}$ \\
	$y^{\text{КНФ}}_g = \overline{\overline{(x_0 \vee x_1 \vee x_2) \cdot (x_0 \vee \bar{x_1} \vee \bar{x_2} \vee \bar{x_3}) \cdot (\bar{x_0} \vee x_1 \vee \bar{x_2} \vee x_3)}}$ \\
	$= \overline{\overline{x_0 \vee x_1 \vee x_2} \vee \overline{x_0 \vee \bar{x_1} \vee \bar{x_2} \vee \bar{x_3}} \vee \overline{\bar{x_0} \vee x_1 \vee \bar{x_2} \vee x_3}}$ \\
	
	\section{Цифровая схема}
	\begin{flushleft}
		Все схемы строились через КНФ.
	\end{flushleft}
	
	\includeimage
	{a_circuit}
	{f} % Обтекание (без обтекания)
	{h} % Положение рисунка (см. figure из пакета float)
	{1.0\textwidth} % Ширина рисунка
	{Схема для светодиода А.} % Подпись рисунка
	\includeimage
	{a_waveform}
	{f} % Обтекание (без обтекания)
	{h} % Положение рисунка (см. figure из пакета float)
	{1.0\textwidth} % Ширина рисунка
	{Сигнал на светодиоде А.} % Подпись рисунка
	
	\includeimage
	{b_circuit}
	{f} % Обтекание (без обтекания)
	{h} % Положение рисунка (см. figure из пакета float)
	{1.0\textwidth} % Ширина рисунка
	{Схема для светодиода B.} % Подпись рисунка
	\includeimage
	{b_waveform}
	{f} % Обтекание (без обтекания)
	{h} % Положение рисунка (см. figure из пакета float)
	{1.0\textwidth} % Ширина рисунка
	{Сигнал на светодиоде B.} % Подпись рисунка
	
	\includeimage
	{c_circuit}
	{f} % Обтекание (без обтекания)
	{h} % Положение рисунка (см. figure из пакета float)
	{1.0\textwidth} % Ширина рисунка
	{Схема для светодиода C.} % Подпись рисунка
	\includeimage
	{c_waveform}
	{f} % Обтекание (без обтекания)
	{h} % Положение рисунка (см. figure из пакета float)
	{1.0\textwidth} % Ширина рисунка
	{Сигнал на светодиоде C.} % Подпись рисунка
	
	\includeimage
	{d_circuit}
	{f} % Обтекание (без обтекания)
	{h} % Положение рисунка (см. figure из пакета float)
	{1.0\textwidth} % Ширина рисунка
	{Схема для светодиода D.} % Подпись рисунка
	\includeimage
	{d_waveform}
	{f} % Обтекание (без обтекания)
	{h} % Положение рисунка (см. figure из пакета float)
	{1.0\textwidth} % Ширина рисунка
	{Сигнал на светодиоде D.} % Подпись рисунка
	
	\includeimage
	{e_circuit}
	{f} % Обтекание (без обтекания)
	{h} % Положение рисунка (см. figure из пакета float)
	{1.0\textwidth} % Ширина рисунка
	{Схема для светодиода E.} % Подпись рисунка
	\includeimage
	{e_waveform}
	{f} % Обтекание (без обтекания)
	{h} % Положение рисунка (см. figure из пакета float)
	{1.0\textwidth} % Ширина рисунка
	{Сигнал на светодиоде E.} % Подпись рисунка
	
	\includeimage
	{f_circuit}
	{f} % Обтекание (без обтекания)
	{h} % Положение рисунка (см. figure из пакета float)
	{1.0\textwidth} % Ширина рисунка
	{Схема для светодиода F.} % Подпись рисунка
	\includeimage
	{f_waveform}
	{f} % Обтекание (без обтекания)
	{h} % Положение рисунка (см. figure из пакета float)
	{1.0\textwidth} % Ширина рисунка
	{Сигнал на светодиоде F.} % Подпись рисунка
	
	\includeimage
	{g_circuit}
	{f} % Обтекание (без обтекания)
	{h} % Положение рисунка (см. figure из пакета float)
	{1.0\textwidth} % Ширина рисунка
	{Схема для светодиода G.} % Подпись рисунка
	\includeimage
	{g_waveform}
	{f} % Обтекание (без обтекания)
	{h} % Положение рисунка (см. figure из пакета float)
	{1.0\textwidth} % Ширина рисунка
	{Сигнал на светодиоде G.} % Подпись рисунка
	
	\includeimage
	{bcd_circuit}
	{f} % Обтекание (без обтекания)
	{h} % Положение рисунка (см. figure из пакета float)
	{1.0\textwidth} % Ширина рисунка
	{Схема для светодиода BSD.} % Подпись рисунка
	\includeimage
	{bcd_waveform}
	{f} % Обтекание (без обтекания)
	{h} % Положение рисунка (см. figure из пакета float)
	{1.0\textwidth} % Ширина рисунка
	{Сигнал на BSD.} % Подпись рисунка
	\includeimage
	{BCD_new}
	{f} % Обтекание (без обтекания)
	{h} % Положение рисунка (см. figure из пакета float)
	{1.0\textwidth} % Ширина рисунка
	{Схема для светодиода BSD(С отрицанием, так как в ПЛИС светодиоды на анодах).} % Подпись рисунка
	
	
	\chapter{Счётчик с коэффициентом счёта 6.}
	
	\section{Счётчик для 6 знаков.}
	\begin{flushleft}
		Построим счётчик для 6 знаков.
	\end{flushleft}

	\includeimage
	{CNT_citcuit}
	{f} % Обтекание (без обтекания)
	{H} % Положение рисунка (см. figure из пакета float)
	{1.0\textwidth} % Ширина рисунка
	{Схема счётчика на D-триггерах для 6 знаков.} % Подпись рисунка

	\includeimage
	{CNT_waveform}
	{f} % Обтекание (без обтекания)
	{H} % Положение рисунка (см. figure из пакета float)
	{1.0\textwidth} % Ширина рисунка
	{Вывод счётчика на D-триггерах для 6 знаков.} % Подпись рисунка

	\section{Делитель частоты.}

	\begin{flushleft}
		Так как на вход будет подан сигнал с частотой 50 МГц, то мы
		ничего не увидим, так как переключение бует слишком быстрым.
		Необходимо собрать делитель напряжения сигнала до частоты в
		50 Гц. Важно будет уточнить, что чтобы проверить схему в 
		Quartus II, сделаем сквозной выход [0], потому что в 
		противном случае сигнал будет нельзя проверить внутри 
		электронной системы так как он будет слишком растянут, в то 
		время перед тем как добавить его в ПЛИС надо будет переключить на [15].
	\end{flushleft}

	\begin{lstlisting}[language=verilog,escapeinside=``]
module VD2(clk, out_pos);
	input clk;
	output reg [19:0]out_pos;

	always @(posedge clk)
	begin
		out_pos <= out_pos + 1'd1;
	end
endmodule
	\end{lstlisting}
	
	\includeimage
	{VD2_20_waveform}
	{f} % Обтекание (без обтекания)
	{H} % Положение рисунка (см. figure из пакета float)
	{1.0\textwidth} % Ширина рисунка
	{Результат деления.} % Подпись рисунка
	
	\chapter{Преобразователь кода, на выходе которого формируется
			 последовательность бинарных чисел соответствующая цифрам
			 студенческого билета.}
	
		\section{Декодер номера билета.}
	\begin{flushleft}
		Для начала построим декодер для билет под номером 20L274.
	\end{flushleft}
	
	\begin{center}
		\begin{tabular}{ |c||c|c|c||c|c|c|c| } 
			\hline
			Символ & $a_0$ & $a_1$ & $a_2$ & $x_0$ & $x_1$ & $x_2$ & $x_3$ \\
			\hline
			2 & 0 & 0 & 0 & 0 & 0 & 1 & 0 \\
			0 & 0 & 0 & 1 & 0 & 0 & 0 & 0 \\ 
			Л & 0 & 1 & 0 & 1 & 0 & 1 & 0 \\ 
			2 & 0 & 1 & 1 & 0 & 0 & 1 & 0 \\ 
			7 & 1 & 0 & 0 & 0 & 1 & 1 & 1 \\ 
			4 & 1 & 0 & 1 & 0 & 1 & 0 & 0 \\  
			\hline
		\end{tabular}
	\end{center}
	
	$x_0^{\text{СДНФ}} = \bar{a_0} a_1 \bar{a_2}$ \\
	$x_1^{\text{СДНФ}} = a_0 \bar{a_1} \bar{a_2} \vee a_0 \bar{a_1} a_2$ \\
	$x_2^{\text{СДНФ}} = \bar{a_0} \bar{a_1} \bar{a_2} \vee \bar{a_0} a_1 \bar{a_2} \vee \bar{a_0} a_1 a_2 \vee a_0 \bar{a_1} \bar{a_2}$ \\
	$x_3^{\text{СДНФ}} = a_0 \bar{a_1} \bar{a_2}$ \\
	
	\begin{flushleft}
		Переведем в 2И-НЕ: 
	\end{flushleft}
	
	$x_0^{\text{ДНФ}} = \overline{\overline{\bar{a_0} a_1 \bar{a_2}}}$ \\
	$x_1^{\text{ДНФ}} = \overline{\overline{a_0 \bar{a_1} \bar{a_2} \vee a_0 \bar{a_1} a_2}} =$ $\overline{\overline{a_0 \bar{a_1} \bar{a_2}} \cdot \overline{a_0 \bar{a_1} a_2}}$ \\
	$x_2^{\text{ДНФ}} = \overline{\overline{\bar{a_0} \bar{a_1} \bar{a_2} \vee \bar{a_0} a_1 \bar{a_2} \vee \bar{a_0} a_1 a_2 \vee a_0 \bar{a_1} \bar{a_2}}} =$
	$\overline{\overline{\bar{a_0} \bar{a_1} \bar{a_2}} \cdot \overline{\bar{a_0} a_1 \bar{a_2}} \cdot \overline{\bar{a_0} a_1 a_2} \cdot \overline{a_0 \bar{a_1} \bar{a_2}}}$ \\
	$x_0^{\text{ДНФ}} = \overline{\overline{a_0 \bar{a_1} \bar{a_2}}}$ \\
	
	\includeimage
	{decoder_20L274_circuit}
	{f} % Обтекание (без обтекания)
	{H} % Положение рисунка (см. figure из пакета float)
	{1.05\textwidth} % Ширина рисунка
	{Схема декодера для студенческого билета 20L274.} % Подпись рисунка
	
	\includeimage
	{decoder_20L274_waveform}
	{f} % Обтекание (без обтекания)
	{H} % Положение рисунка (см. figure из пакета float)
	{1.0\textwidth} % Ширина рисунка
	{Вывод декодера для студенческого билета 20L274.} % Подпись рисунка
	
	\includeimage
	{CNT_citcuit}
	{f} % Обтекание (без обтекания)
	{H} % Положение рисунка (см. figure из пакета float)
	{1.0\textwidth} % Ширина рисунка
	{Схема счётчика на D-триггерах для 6 знаков.} % Подпись рисунка
	
	\includeimage
	{CNT_waveform}
	{f} % Обтекание (без обтекания)
	{H} % Положение рисунка (см. figure из пакета float)
	{1.0\textwidth} % Ширина рисунка
	{Вывод счётчика на D-триггерах для 6 знаков.} % Подпись рисунка
	
	\begin{flushleft}
		Как~видно~из~графика~\ref{img:CNT_waveform}, как только значение доходит
		до 1 1 0 происходит сброс значений. Пострим финальную схему для вывода
		номера билета.
	\end{flushleft}
	
	\section{Общая схема.}
	\includeimage
	{counter_circuit}
	{f} % Обтекание (без обтекания)
	{H} % Положение рисунка (см. figure из пакета float)
	{1.0\textwidth} % Ширина рисунка
	{Схема вывода билета.} % Подпись рисунка
	
	\includeimage
	{counter_waveform}
	{f} % Обтекание (без обтекания)
	{H} % Положение рисунка (см. figure из пакета float)
	{1.0\textwidth} % Ширина рисунка
	{Вывод вывода билета.} % Подпись рисунка

	\begin{flushleft}
		На~временных~диаграммах~\ref{img:counter_waveform}, значения на элементе 
		CNT приходят с вывода TEST элемента VD2\_20, чтобы показать корректность 
		работы счетчика.
	\end{flushleft}

	\includeimage
	{counter_pin}
	{f} % Обтекание (без обтекания)
	{H} % Положение рисунка (см. figure из пакета float)
	{1.0\textwidth} % Ширина рисунка
	{Подключения пинов на ПЛИС Cyclone IV E - EP4CE6F17C8.} % Подпись рисунка
	\begin{flushleft}
		Произошла проблема с запуском на ПЛИС, все не корретно отображается. 
		Исправляем ошибки.
	\end{flushleft}
	
	\section{Исправление ошибок.}
	
	\begin{flushleft}
		Вывод билета на экран был не корректный, символы были в перемешку, было 
		решено добавить дополнительный элемент SUB, который выставляет символы в 
		правильном порядке.
	\end{flushleft}
	
	\includeimage
	{SUB}
	{f} % Обтекание (без обтекания)
	{H} % Положение рисунка (см. figure из пакета float)
	{1.0\textwidth} % Ширина рисунка
	{Преобразователь.} % Подпись рисунка
	
	\begin{flushleft}
		Подставим элемент в схему:
	\end{flushleft}

	\includeimage
	{counter_circuit_new}
	{f} % Обтекание (без обтекания)
	{H} % Положение рисунка (см. figure из пакета float)
	{1.0\textwidth} % Ширина рисунка
	{Обновленная схема.} % Подпись рисунка
	
	\begin{flushleft}
		Вывод:
	\end{flushleft}

	\includeimage
	{20L274}
	{f} % Обтекание (без обтекания)
	{H} % Положение рисунка (см. figure из пакета float)
	{1.0\textwidth} % Ширина рисунка
	{Вывод моего билета 20Л274.} % Подпись рисунка
	
	\begin{flushleft}
		P. s. П = Л = L
	\end{flushleft}
\end{document}





























